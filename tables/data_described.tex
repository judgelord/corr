\paragraph{Response Rates to Our FOIA Requests}
As of August 2020, all departments except for the State Department have provided records, though the majority of records from the Departments of Defense and Energy are still being processed, and the majority of Department of Justice components have not yet released records to us (see Table \ref{responserates}). 
Some components provided records that duplicated records from the Secretary's office, for example in the Department of Agriculture where several such component records were dropped. As for independent agencies, we are waiting on records from the SEC, FLRA, CFPB, CIA, and Appalachian Regional Commission. 
The remaining 28 independent agencies have provided records, though some are still in the process of reviewing and releasing additional records. 
Of these, 18 have been sufficiently cleaned, coded, and linked with other data source for inclusion in this analysis. 
The large amount of data yet to be received will allow out-of-sample tests of the present analysis. In all we have filed \input{../data/FOIA_requests}\unskip, yielding input{../data/n}\unskip observations.

About half of the responsive agencies are left-censored between 2007 and 2013. 
Left censoring arises from either document retention cycles (offices that are diligent about discarding documents), or document loss and changing systems (offices that are bad at keeping documents). 
Either of these may correlate with changes in an agency's salience, for example, due to changes in party control.
The most contacted and controversial agencies tend to keep higher quality records.
This might introduce bias toward older records being about policy, but we do not see evidence of such bias in our data. 

 \paragraph{Variation in Responses to Identical FOIA Request} 
 Responses to our requests varied significantly. 
 Most agencies offered logs of congressional correspondence, which record a date, sender, and summary of each contact. Logs generally include any written requests, as well as many phone and email records. 
For example, Between May 2015 and December 2017, the Department of Justice Office of Administrative Law Judges received 132 emails, 109 telephone calls, and only 54 letters. 
Between 2007 and 2017, the Postal Regulatory Commission received 100 emails, 30 faxes, 173 letters, 118 calls. In this paper, we use ``contacts'' and ``letters'' interchangeably to refer to all modes of correspondence. \

Small agencies or regional offices had staff search their email history or provided hand-written records that we had transcribed. 

Department Secretary offices generally queried a correspondence tracking database designed to track all correspondence, but our FOIA requests to sub-departmental components almost always recovered additional records of communication that was not in central databases. 
As one central office FOIA officer put it ``Legislative Affairs is supposed to be the front door for the department, but if somebody knows somebody, well...'' (personal communication, Feb. 21, 2018). 
Because of such idiosyncratic relationships, capturing patterns of correspondence that ``go around'' a Department Secretary's office is key to avoiding erroneous inferences about legislator behavior. 
For example, when chairs of the Homeland Security committee wrote about immigration enforcement issues, they almost always contacted the Department of Homeland Security (DHS) office of the Executive Secretary, but, at the same time, the Immigration Customs Enforcement (ICE) component of DHS directly received thousands of contacts from a different set of legislators. 
