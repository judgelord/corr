%!TeX program = pdflatex
\documentclass[12pt]{article}
\usepackage{lmodern, amssymb,amsmath, graphicx, hyperref}
% === bibliography package ===
\usepackage{natbib}
% \usepackage[colorinlistoftodos, prependcaption]{todonotes} % to use \todo 
\usepackage[margin=1in]{geometry}
\usepackage{csquotes}
\usepackage{lscape}
\usepackage{setspace}
\usepackage{ulem}
\hypersetup{
  colorlinks=true,
  linkcolor=blue,
  filecolor=magenta,    
  urlcolor=cyan,
  citecolor = black
}
\newenvironment{tight_itemize}{
\begin{itemize}
 \setlength{\itemsep}{0pt}
 \setlength{\parskip}{0pt}
 }{\end{itemize}}
\urlstyle{same} % don't use monospace font for urls

 \title{The Effects of Shifting Priorities and Capacity on Policy Work and Constituency Service: Evidence from a Census of Legislator Requests to U.S. Federal Agencies}
  \author{Devin Judge-Lord\thanks{Postdoctoral Fellow, Harvard University, DevinJudgeLord@fas.harvard.edu.}\and Justin Grimmer\thanks{Professor, Stanford University and Senior Fellow, Hoover Institution} \and Eleanor Neff Powell\thanks{Booth Fowler Associate Professor, University of Wisconsin-Madison}}
  \date{\today}

\usepackage{booktabs} % To thicken table lines
\usepackage{multirow}
\bibliographystyle{apsr}

% for modelsummary
\usepackage{siunitx}
\newcolumntype{d}{S[input-symbols = ()]}


\begin{document}

\maketitle

\begin{abstract}

\noindent When elected officials gain power, do they use it to provide more constituent service or affect broad public policies? Answering this question informs debates over the effects of legislator capacity, term limits, and institutional power on political representation. We distinguish two countervailing effects of increased institutional power. First, as elected officials gain power, they allocate relatively more effort to policy over constituency service. Second, institutional power provides additional resources and therefore increases their overall capacity. To assess the extent to which these countervailing effects of institutional power affect behavior, we assemble a massive new database of  \input{../tables/n} Congressional requests to federal agencies between 2007 and 2018 obtained through \input{../tables/foia_requests} FOIA requests, a near census of departments, agencies, and sub-agencies. We find that most legislator contacts with the bureaucracy are constituency service, regardless of institutional position and tenure. Leveraging variation within legislator-agency pairs, we show that legislators prioritize policy work as they gain power and experience, but increasing overall capacity enables them to maintain levels of constituency service. Consistent with our theory that service depends on capacity, we show that new legislators do less policy work and constituency service than their more senior colleagues. In a series of robustness checks, we show that our findings are not the result of exogenous variation in constituent demand. Rather than long-serving and powerful elected officials diverting attention from their district,  their increased capacity enables them to maintain levels of constituency service, even as they prioritize policy work.
\end{abstract}

Word Count: 12375

\newpage

%\tableofcontents

\doublespacing

\section{Introduction}

One of the oldest traditions of representation in American politics is constituency service---how elected officials help channel and articulate individual constituents' demands to government institutions. Constituency service is when members of Congress ``provid[e] help to individuals, groups, and localities in coping with the federal government" \citep{Fenno1978}.\footnote{This tradition of constituency service can be traced back to the first congresses when constituents sought assistance with Revolutionary War pensions \citep{Eckman2017}. Constituents seek assistance with various topics from Social Security, Disability, and Veterans Benefits to Citizenship Applications to pollution and employment discrimination complaints.} Advocating on behalf of their constituents to federal agencies is a crucial part of a modern legislator's job, and its growth has been used to explain incumbency advantage \citep{King1991}. Yet despite the centrality of constituency service in theories of congressional representation, constituency service remains one of the least understood congressional activities.\footnote{As our data show, modern constituency service encompasses much more than the oft-cited examples of helping constituents with federal benefits. Members also serve constituents by seeking to help intercede on their behalf with a host of lesser-known federal agencies and advocate for state or local governments or nonprofits who apply for federal grants, permits, or disaster recovery funds.} Over thirty years ago \citet*{CainFerejohnFiorina1987} began their seminal book \textit{The Personal Vote: Constituency Service and Electoral Independence} with that same observation, and much of what we know empirically today about constituency service comes from their surveys of legislators, legislative staff, and constituents. The disproportionate academic focus on legislative activities has left unanswered long-standing questions about how legislators balance the pursuit of legislative goals with constituency service provision. Likewise, we know little about how levels of constituency service vary across legislators. 
%\todo{ELLIE: any sources citing the lack of constituent service lit in the past 35 years?}



%Is variation provision of constituency service caused by shifts in legislators' capacity or shifts in their priorities throughout their career? 

This paper examines how increasing power in Washington affects the provision of constituency service. On the one hand, formal models of accountability \citep{AshworthBuenodeMesquita2006} imply that if constituency service enables elected officials to demonstrate competence to their constituents, then increased institutional power and capacity will result in increased levels of service to constituents. This increase in service would occur as legislators use the increased resources that come with better committee assignments to put more effort into constituency service to satisfy the primary goal of reelection. Experienced incumbents may also have an advantage over challengers if newly elected officials incur start-up costs that reduce their capacity to provide constituency service. The prediction from these models is that the level of service will rise as elected officials gain institutional power, hire staff, and establish systems for soliciting and handling constituency service opportunities.

On the other hand, we might expect that as legislators spend more time in Washington and gain prestige, they become more focused on general policy work and less attentive to their district and constituents. This dynamic is central to theories of representation that examine legislators' careers and the tradeoffs they face when acquiring power. As legislators acquire power in the institution, it is often asserted that they catch ``Potomac fever'' and devote less attention to constituents back in the district \citep{Fenno1978}. Some assume that the effect of shifting priorities is large enough to cause long-serving legislators to lose touch with their district and become poor representatives. Such reasoning is the primary justification for proposed term limits. Related arguments are common also in the popular press \citep{Edwards2005} and evoked in rallying cries to ``drain the swamp'' of legislators focused on the Washington elite \citep{Rosenblatt2016}.

% Increasing capacity and shifting priorities have countervailing effects on the volume of constituency service. 
We test these competing expectations with a new and massive data set of constituency service: a near census of legislator contacts with federal agencies from 2007 to 2018. We build on recent work using data on congressional correspondence that has yielded important findings regarding the policy strategies of cross-pressured legislators \citep{Ritchie2017}, distributive politics \citep{MillsKalafHuges2015}, descriptive representation \citep{LowandeRitchieLauterbach2018}, and the role of ideology in congressional oversight \citep{Lowande2018JOP}. Committee oversight relationships help explain which legislators engaged in policy advocacy \citep{Ritchie2017,Lowande2018JOP}.
To date, this emerging scholarship has focused on policy work. 
Adding to this work, our theory and research design focus squarely on constituency service.


Given the difficulty in collecting these data, previous work has been restricted to small subsets of agencies and thus a small subset of policy domains. Our larger data set enables us to comprehensively test how the behavior of legislators shifts as they gain institutional power and ensures that our conclusions are not dependent upon the subset of the executive branch that we examine. To assess absolute and relative shifts in legislator contacts to agencies, we hand code the content of \input{../tables/n} requests as policy work or constituency service. Doing so also yields many illuminating descriptive facts. For example, over 80\% of requests are made on behalf of constituents, and less than 20\% are focused on more general policy work. 



% WE FIND SUPPORT FOR BOTH DYNAMICS 

Using this new data set and a robust research design, we find evidence for both of the countervailing effects of institutional power that we theorize. Crucially, we find that the magnitude of the effect of increasing capacity on constituency service offsets the effect of shifting priorities toward policy work, such that legislators provide no less constituency service as they gain power in Washington. Members of Congress increasingly prioritize policy work as they gain institutional power, but the capacity they gain allows them to increase their volume of policy work without decreasing the volume of constituency service.

Using a within-legislator-agency pair difference-in-differences design, we show that more power in Congress---as measured through the acquisition of Committee power---causes legislators to make more overall contacts with federal agencies. For example, using our preferred specification, we find that becoming a committee chair causes a 24 percent increase in contacts with federal agencies. Consistent with capacity affecting legislators' constituency service, we also find that new legislators provide less constituency service and do less policy earlier in their career than later in their career and that districts receive less constituency service overall in the first two years after electing a new representative.  



At the same time, we find evidence that legislators prioritize policy work as they acquire institutional power. Using a within-legislator difference-in-differences design, we show that legislators increase the ratio of policy work to constituency service as they gain institutional power. When legislators become committee chairs, they increase the share of policy-related contacts by seven percentage points. Becoming a ranking minority member causes a three percentage point shift towards policy work.

%Despite powerful legislators' shift towards policy, acquiring more power does not affect the number of constituency service-focused requests legislators make for their constituents. The expanded capacity of more powerful legislators compensates for the shift in priorities. 

These findings are robust; they are not the result of exogenous variation in constituent demand, differences across districts, or differences across legislators. Our research designs limit the influence of any potential variation in constituent demand by leveraging within-district comparisons. Moreover, through a series of robustness checks and additional analyses, we find no evidence that constituents shift demands to more established offices when a new representative is elected. When House members lose an election, there is no corresponding increase in constituency service requests from the state's Senate delegation. 

Our work informs both the \textit{increased capacity theory} and the \textit{shifting priorities theory} of representation. While we find evidence that legislators' attention shifts towards policy as they remain in Washington, the amount of constituency service remains constant. As a result, constituents do not face a tradeoff between powerful and attentive representatives.   If constituents value constituency service, then they are no worse off when legislators acquire power. But if constituents also value the policy work of their representatives, then constituents' are better off with legislators with more power. 

While this study does not aim to examine the effects of term limits, our results have implications for debates about term limits. Advocates for term limits assert that powerful career politicians become alienated from their constituents. We show, however, that even as legislators gain power, they remain focused on providing constituency service. In fact---in contrast to arguments from these organizations---the biggest decrease in constituency service occurs as new legislators encounter start-up costs and provide fewer overall contacts with bureaucratic agencies. Under the pattern we find, the constant turnover that term limits induce would cause a sharp decrease in the volume of work from legislative offices. Empirical work finds that term-limited legislators who can
no longer seek reelection sponsor fewer bills, are less productive on committees, and are absent for more floor votes \citep{Fouirnaies2021}. Our findings suggest that the newly-elected members who replace term-limited legislators would initially be less productive as well. 
% other term limit papers include Besley and Case 1995; List and Sturm 2006; Alt, Bueno de Mesquita, and Rose 2011; Ferraz and Finan 2011


This paper proceeds as follows. Section \ref{s:theory} explains divergent predictions about how legislators' behavior will be affected by institutional position and tenure in Congress. Section \ref{s:data} explains our data collection process and provides basic summary statistics. Section \ref{s:results} shows that legislators with more experience and better committee assignments do more constituency service, even as they shift their priorities toward policy work. Section \ref{s:demand} provides robustness checks for our results and explores alternative explanations. Section \ref{s:conclude} highlights implications of these findings for theories of legislative behavior.





\section{Do Experience and Power Increase or Decrease Constituency Service?} \label{s:theory}



How elected officials balance their work on broad policy goals and delivering particularistic service to their constituents and district presents a significant tension for representation. Legislators' experience and acquisition of power in Congress likely affect how they balance delivering service to constituents and working on broader legislation. But comparative statistics from formal models of accountability have divergent predictions of how increased power and experience will affect legislators' attentiveness to the district. Building on multi-task models of representation \citep{AshworthBuenodeMesquita2006, gordon2009advantages}, we explain why we might expect increased experience and power to either increase or decrease the levels of constituency service legislators provide depending on the relative magnitudes of the effects of increasing capacity and changing priorities.

%But there are competing theoretical expectations of how increased experience and power will affect how much constituency service legislators provide constituents and the kind of service they provide. On the one hand, formal models of constituency service predict that increased experience and power will cause legislators to bolster the amount of constituency service provided. Related to these models is the expectation that a new legislator will face high start-up costs as they hire new staff and establish the procedures that will effectively handle the constituency service caseload. After a newly elected official incurs these costs, constituency service would increase. On the other hand, concerns about legislators becoming fixated on careers in Washington lead to an expectation that power in Washington comes at the expense of constituencies, as legislators allocate more of their resources and efforts to influencing national-level legislation. We explain the logic of these different theoretical expectations in this section and explain why it is unclear how legislators with greater experience or prestige will alter their constituency service to constituents.  




\subsection{Why Experience and Power Could Increase Constituency Service: Increasing Capacity Hypothesis}

%%THERE IS AN ABSOLUTE AND RELATIVE ARTICULATION OF THE PRIORITY HYPOTHESIS

%% THE ABSOLUTE IS THAT THE SERVICE WILL GO DOWN AS LEGISLATORS LOSE THEIR ATTENTIVENESS. 
%% THE RELATIVE IMPLICATION IS A SMALLER SHARE OF THE LEGISLATORS OVERALL EFFORT. THE THEORY IS LARGELY SILENT.  

As elected officials garner more experience in Congress, one prediction from formal models of accountability is that legislators will provide more constituency service because their capacity to do so increases. An influential set of formal theory papers argue that voters are fundamentally engaged in a screening task: attempting to identify elected officials who are competent and effectively deliver representation to the district \citep{AshworthBuenodeMesquita2006, gordon2009advantages}. Under this model of representation, constituency service helps reelection-minded legislators increase their chance of reelection if they can exceed constituents' expectations of the level of service that another candidate would provide.  

Critically, constituents' demands for service do not go away, even as legislators acquire power in the institution. Even if constituents appreciate their representative's power over policy, they still expect their elected officials to be attentive to the district and demonstrate their competence with constituency service. Moreover, if constituency service helps with reelection, legislators may invest in \textit{creating} demands (for example, by advertising constituency services) that they can then meet. If this intuition is correct, elected officials should continue to provide constituency service proportional to their resources and capacity to do so. 

All else equal, these models predict that as a legislator's resources and capacity increase, they will increase their level of constituency service \cite[Proposition 1]{AshworthBuenodeMesquita2006}. Research suggests that the low-level of congressional capacity in the modern Congress serves as a major constraint on Congress's ability to function \citep{LaPira2020}.\footnote{This decline in congressional capacity and the limits low capacity poses on congressional activity has been a focus of substantial scholarly attention and concern in recent years culminating in an edited volume with contributions from 28 scholars \citep{LaPira2020}.} 
Experience in office and institutional power may increase an individual legislator's capacity in many ways. Because many of these mechanisms are observationally equivalent, we focus on capacity in general and three general mechanisms by which experience and institutional power may affect behavior: increased resources, increased organizational capacity, and an increased likelihood of success when making a request. 

\paragraph{Increased Resources} As legislators acquire more institutional power, they usually gain more resources. For example, becoming chair of a Congressional committee provides legislators with a better ability to direct committee staff and a larger budget. New committee chairs often bring in new staff who are loyal to their priorities \citep{Fox1977, DeGregorio1995}. Even if these resources are earmarked for policy work, they can increase the resources legislators have available for constituency service. This is because legislators with more access to committee resources can use those resources for policy work, freeing their personal office resources for constituency service.\footnote{While the role of congressional staff in a legislator's productivity has often been overlooked in congressional studies, a growing body of research has documented the vital role they play in accomplishing member's goals \citep{HertelFernandez2019, MontgomeryNyhan2017, McCrain2018, CrossonEtAl2020, Reynolds2020}.}
%% TODO: ADD COMMITTEE STAFF LIT HERE 

\paragraph{Increased Organizational Efficiency} Better organized legislator offices are more able to help constituents navigate the federal bureaucracy. On average, experienced legislators should have better systems that allow them to make more constituency service requests than new legislators, who face ``start-up" costs that decrease the number of requests they make to agencies. When a new legislator is elected to office, they face a substantial administrative burden. Not only must that legislator hire new staff and open district offices. The legislator also needs to establish protocols, priorities, and procedures in their office for handling constituency service requests. They also lack many of the ``standard" procedures that more established offices use to handle particular problems. In terms of formal models, office organization increases the resources the legislator has available
\citep{AshworthBuenodeMesquita2006}. As legislators build an office and establish protocols, these start-up costs should subside, allowing legislators to make full use of the resources available to their office. As a result, we would expect new legislators to make fewer constituency service requests, but only for the time that it takes to establish their constituency service operation within the office.   
% ADD OFFICE STAFF LIT HERE 
% For example, \citet{Cottle2022} finds that committees led by more experienced legislators and more experienced senior staff are more productive. % In slack
% \citet[p. 28]{Hall1996}  "faced with the press of excessive obligations and the frequent prospect of needing to be two places at once, members have responded by relying increasingly on staff."



\paragraph{Increased Likelihood of Success} 

In fulfilling statutory missions, agencies must prioritize resources and use broad discretion, not only in processing visa, permit, and grant applications but in regulating private entities' compliance with environmental, health, and labor laws and much more. 
% For a vast range of demands involving public or private actors, a federal agency will often be able to help if it prioritizes that request over others. %\todo{CITE}
Legislators are in a position to influence these decisions.
As public servants, agency staff may assign special importance to the requests of elected officials. For example, bureaucrats often tag congressional correspondence as ``VIP," and agency protocols often require faster response deadlines and higher signature levels. %\todo{CITE GUIDANCE}
% Agencies also have strategic reasons to meet legislators' requests. Ad hoc review of a social security disbursement, visa application, or pipeline permit may be inefficient and diverge from protocol. Still, it is a small price to pay if it could help the agency gain a small advantage in securing desired authorizations and budgets. %\todo{CITE}
Bureaucrats have incentives to build relationships and reputations that enhance their standing among members of Congress and those who have their ear, and they actively do so \citep{Carpenter2001}. In short, complying with legislator requests may help agencies achieve their own goals. If an agency aims to grow its coalition of political supporters, we would expect them to accommodate congressional requests frequently.

As legislators become more powerful, agencies may be more responsive. More powerful legislators can more easily alter an agency's budget or create additional work through Congressional hearings. As a result, agencies may prioritize the service requests from the most powerful members of Congress. % CITE?
For example, \citet{Lowande2018JOP} finds that agencies systematically prioritize the requests of majority party legislators.  \citet{MillsKalafHuges2015}  find that the Federal Aviation Administration was less likely to grant the requests of junior members of Congress .\footnote{Because this paper focuses on legislator behavior rather than agency behavior, it is sufficient that more powerful members of Congress occasionally believe that they are more likely to get a response when allocating their efforts. Regarding agency behavior, there is active scholarly debate over the extent to which agencies respond differentially to more powerful legislators. In contrast to canonical theories set out by  \citet{Arnold1979},  \citet{BerryBurdenHowell09} find no evidence that committee membership shaped the distribution of executive-branch spending. \citet{RitchieYou2018} find that legislator requests influenced Department of Labor decisions, but notably, this influence was not correlated with oversight committee membership. Likewise   \citet{MillsKalafHuges2015} even find that the Federal Aviation Administration was \emph{less} likely to grant the requests of members of their authorizing committee, which they attribute to the agency punishing committee members for recent budget cuts, a different form of strategic response to individual members theorized by \citet{Arnold1979}.}
An increased likelihood of response may provide legislators with the opportunity to demonstrate their effectiveness to their constituents. This increases the marginal return on making requests, providing more prestigious legislators with more incentive to make those requests \citep{CainFerejohnFiorina1987}.
A related literature similarly finds that seniority and committee membership affect the distributive politics of earmarks \citep{Lazarus2010}.

Because this third mechanism operates as a multiplier on institutional power and organizational capacity, it is observationally equivalent to the first two mechanisms for our present analysis below.
In short, the observable implications of theories emphasizing the effects of capacity and resources are that legislators with more experience and more powerful institutional positions like committee chairs will do more constituency service work.



\subsection{Why Experience and Power Could Decrease Legislators' Constituency Service Efforts: Shifting Priorities Hypothesis}

A different set of expectations for how legislator behavior may change in response to increased experience and power is present in the political science literature on Congressional careers tracing back to \citet{Fenno1973}. Legislators do not only care about reelection; they also have policy goals. As legislators spend time and gain power in Congress, the marginal impact of the resources they allocate to policy work increases. Powerful legislators thus have incentives to shift their attention to policy work. If this theory is right, as legislators gain power and experience, they should increase the ratio of policy work to constituency service.

As legislators spend more time in Washington, they may become detached and alienated from their district. \citet{Fenno1978} documents that some members of Congress catch ``Potomac fever." While newly elected officials may remain primarily focused on reelection, senior elected officials may prioritize other goals. As they acquire power, it is often asserted that members of Congress ``go Washington'' and devote less attention to constituents back in the district. It seems intuitive that as they spend more time in Washington and attain more influential institutional roles in Congress, legislators might focus on other priorities resulting in less attention paid to constituents.  \citet{Fenno1973} identifies five goals: reelection, power in the House, good public policy, a career beyond the House, and private gain. As Fenno describes in \textit{Congressmen In Committees}, different institutional positions (congressional committees) allow legislators to advance different goals. 

If the shifting priorities hypothesis is right--that legislators shift attention from their district to policy work--we expect that legislators provide \textit{relatively less} constituent service compared to policy work as they gain experience and power. The ``Potomac fever" concern is that the magnitude of the effect of shifting attention and priorities toward policy goals is large enough to swamp any change in their capacity as legislators gain experience and power. Thus, if the strong (``Potomac fever") version of the shifting priorities hypothesis is true, we should find that legislators provide \textit{less} constituent service in absolute terms. 

A parallel argument about legislator behavior emerges in public advocacy for institutional reforms, particularly from those who advocate for term limits. These activists argue that elected officials who are repeatedly re-elected to Congress become detached from their district. For example, Ted Cruz (R-TX) argued in favor of term limits in a Senate hearing, stating that the politicians at the time of founding traveled to Washington and then planned to return to their district. In its place, Cruz argued that ``Today, members of Congress aren't doing that. Instead, far too many of our politicians come to Washington to stay." Alexandria Ocasio-Cortez made a similar indictment against Joe Crowley in her successful primary challenge. She would regularly remind audiences that Crowley had ``been there for 20 years" and then ask, ``What has this power been used for? It's not being used for us." %https://www.termlimits.com/senator-ted-cruz-hearing/ https://www.thenation.com/article/alexandria-ocasio-cortez-fights-power/

Formal models of representation predict that as legislators prioritize policy work in Washington, the ratio of policy work to constituency service will increase. All else equal, they will then provide lower levels of constituency service to their district. For example, in the model in \cite{AshworthBuenodeMesquita2006}, this would occur as legislators place a lower priority on constituency service. If legislators shift their priorities through their careers, more experienced legislators will allocate their staff to policy work, and legislators who acquire more institutional power would focus their time on law-making and policy oversight rather than constituency service work. Likewise, if members shift their \textit{attention} from the district to a career in Washington (in Congress or after), their relative level of attention to constituent issues will also decrease. 


%\subsection{Relative vs. Absolute Shifts and Overall Provision of Service} 
\subsection{The Countervailing Effects of Increasing Capacity and Shifting Priorities on Constituency Service}

Legislators experience career shifts that may simultaneously increase their resources and decrease the relative priority they place on constituency service. The net effect of these countervailing shifts on the levels of constituency service they provide depends on the relative size of legislators' resource increase compared to the size of the shift in their priorities. Increased capacity may offset a shift in focus away from constituency service as a legislator gains power. If this occurs, we would expect the ratio of policy work to constituent service to increase as legislators gain experience and power, but we would also expect the absolute level of constituency service to stay the same or increase. Alternatively, if the effect of increased capacity is smaller or the effect of shifting priorities larger, absolute levels of constituency service may decline as a legislator gains power.


Table \ref{t:theory} and Figure \ref{f:2x2-fig} show expected changes in the absolute volume of constituency service and the ratio of policy work to constituency service due to \textit{changes in capacity} and \textit{shifting priorities}. The top-left cell of Table \ref{t:theory} shows our expectations if both mechanisms affect behavior \textit{and} the magnitude of the effect of increased capacity is large enough to overcome the countervailing effect of shifting priorities (any outcome in the upper, darker-shaded region of Figure \ref{f:2x2-fig}, subfigure 4).

{\renewcommand{\arraystretch}{1.1}% THIS PADS ALL TABLES

\begin{table}[]

\caption{Divergent Predictions for the Change in the Levels of Constituency Service and Policy Work as Legislators Gain Experience and Power}\label{t:theory}

\begin{tabular}[t]{p{.15\linewidth}|p{.20\linewidth}|p{.32\linewidth}|p{.33\linewidth}|}

\multicolumn{2}{l}{\multirow{2}{*}{}} & \multicolumn{2}{c}{Increasing Capacity Hypothesis} \\ \cline{3-4}

\multicolumn{2}{l|}{}    &  Increase in Capacity  &   No Change in Capacity \\ \cline{2-4} 

\multirow{4}{1.8cm}{Shifting Priorities Hypothesis}  &   \multirow{2}{3cm}{Priority Shifts to Policy Work}   &  Level of Service: 0 or $\uparrow$  &  Level of Service: $\downarrow$  \\ 

& &  Ratio of $\frac{Policy}{Service}$: $\uparrow$   &   Ratio of $\frac{Policy}{Service}$: $\uparrow$  \\ \cline{2-4}

 &  No Change    &  Level of Service: $\uparrow$  & Level of Service: $0$ \\ 

 & in Priorities &   Ratio of $\frac{Policy}{Service}$: $0$  & Ratio of $\frac{Policy}{Service}$: $0$\\ \cline{2-4}

\end{tabular}

\end{table}


Figure \ref{f:2x2-fig} formalizes the potential outcomes implied by our theory in order to clarify the conditions under which constituency service will increase or decrease as an elected official's capacity and priorities change. 
Figure \ref{f:2x2-fig} visualizes potential outcomes for a range of potential changes in capacity extending up to 150\% of some baseline level of capacity. For simplicity, we set baseline capacity at 100 (e.g., 100 contacts with federal agencies per year) and the baseline ratio of constituency service to policy work at 80:20 (80\% constituency service).
An elected official's level of policy work, $x$, depends on their overall capacity, $c$, and relative priority for constituency service versus policy work, $p \in [0, 1]$ (that is, the share of contacts that are constituency service rather than policy work) such that $x_i = c_i (1-p_i)$.
An elected official's level of constituency service, $y$, likewise depends on their overall capacity and priorities such that $y_i = c_i p_i$.
For any given level of capacity, $y_i = c_i-x_i$ specifies a line of possible divisions of capacity between policy work and constituency service. Increasing capacity pushes this "capacity frontier" line to the upper right (Figure \ref{f:2x2-fig}, subfigure 1). Where on this line a legislator exists at any point in time depends on their relative priority for policy work and constituency service (Figure \ref{f:2x2-fig}, subfigure 2). If capacity and priorities shift simultaneously, priorities can shift toward policy while levels of constituency service are maintained or even increase (Figure \ref{f:2x2-fig}, subfigure 3). When shifting priorities and increasing capacity occur simultaneously, the relative magnitude of these two effects determines whether constituency service will increase or decrease (Figure \ref{f:2x2-fig}, subfigure 4).  
When $c_1p_1 > c_2p_2$, constituency service decreases. 
When $c_1p_1 < c_2p_2$ constituency service increases. 
When $c_1p_1 = c_2p_2$, there is no change in constituency service. 

\begin{figure}
\centering
\caption{The Countervailing Effects of Increasing Capacity and Shifting Priorities on Constituency Service} \label{f:2x2-fig}
\includegraphics[width = \textwidth]{figs/2x2-fig-1}
\end{figure}






\subsection{Alternative Explanation for Changes in Constituency Service: Constituent Demand}

We aim to understand how gaining power and experience in Washington affects how legislators serve constituents. But often, legislators depend on constituents to ask for help navigating the federal bureaucracy. An alternative explanation for why legislators' rates of constituency service provision vary is that they receive differing numbers of requests from their constituents for reasons that are not a result of legislators' capacity and experience. While such variation in constituent demand is substantively interesting, such an explanation for constituency service provision does not inform debates about the extent to which legislators' priorities change with their time in office and institutional position. 

Of course, constituent demands inform which agencies legislators contact. For example, some districts contain groups---such as veterans or social security recipients---who request particular kinds of constituency service from their representatives. Our research design limits the influence of this kind of variation in constituent demand by examining how an individual legislator's rates of contact change within particular agencies. By looking at the same member representing the same constituents in the same district to the same agency over time, we limit the extent to which differing constituent populations could interfere with our results. 

Legislators may also use their official resources to encourage requests from constituents for help navigating the federal bureaucracy through workshops, newsletters to constituents, social media posts, and even stories in local papers. 
Such constituent outreach may even be a primary way that constituencies discover that their elected officials can help with problems they may have with the bureaucracy. If legislators use increased staff budgets or organizational capacities to solicit constituent requests, constituent demands may increase as legislator power increases \citep{CainFerejohnFiorina1987}.  %Increases in constituents' demands may thus reflect elected officials' capacity and efforts to solicit demand for constituency service.
This is entirely consistent with our theory that increased power and capacity enable legislators to provide constituency services as well as policy work. Our theory and tests do not require that legislators allocate resources to soliciting constituency service demand, but if they do, the underlying cause is shifts incapacity, not some exogenous shift in constituent demand that could confound our analysis. Thus, this form of constituent demand is not a problem for our analysis. Indeed constituent service outreach may be a key mechanism for the capacity effects we theorize.

A more challenging form of constituent demand could exist if constituents redirect their requests towards legislators who they expect to be more powerful. Constituents might expect that more powerful legislators could more effectively provide constituency service and, as a result, direct their demands towards those legislators. If constituents strategically redirect their demands for help from representatives that lost a chair position to representatives that gained a chair position, this could partially confound our analysis. More realistically, if constituents redirect requests for help away from new legislators toward longer serving legislators, we might observe increases in demand targeted at more experienced legislators of a delegation whenever another more experienced member of their state's delegation is replaced by a less experienced legislator.  
To address these sorts of concerns, in Section \ref{s:demand} we focus on a series of robustness checks to rule out alternative constituency demand explanations. We find no evidence of requests spilling over to more experienced and powerful members within a state delegation.   

%%%%%%%%%%%%%%%%%%%%%%%%%%%
%% DATA %%%%%%%%%%%%%%%%%%%
%%%%%%%%%%%%%%%%%%%%%%%%%%%

\section{A Census of Legislator Requests to Federal Agencies} \label{s:data}
To assess how experience and power affect constituency service, we filed  \input{../tables/foia_requests} Freedom Of Information Act (FOIA) requests with all federal departments, agencies, and sub-agencies for all records of incoming communication from members of Congress between January 1, 2007, and the date of our request.\footnote{In addition to our initial requests, collecting these data included over a thousand follow-up and clarification emails, dozens of hours on the phone with FOIA officers, and nearly 100 appeals of incomplete records or inappropriate denials, including multiple cases where we pursued and won orders from judges requiring compliance our request. By rigorously pursuing a census of records, we limit any response bias that may exist in more easily-obtained samples.} Between February 2017 and February 2021, we received data on \input{../tables/n} instances of members of Congress contacting federal agencies. We focus on requests made from 2007-2018, resulting in a data set of  \input{../tables/n2007-2018}contacts.\footnote{Some agencies did not provide records for the full span of years. Our models include legislator-by-agency fixed effects to account for any left censoring, ensuring that our comparisons leverage variation within each agency, limiting the opportunity for left-censoring to affect our results.} 

Our data represent a near census of requests to federal departments, agencies, and sub-agencies. We received records from every department other than the Department of State,\footnote{The Department of State has a notorious FOIA backlog of approximately 10,500 cases. The FOIA office expects to fill our May 2018 request in 2024.} and most independent agencies, commissions, boards, executive offices (e.g., the Council on Environmental Quality and U.S. Trade Representative), and pseudo-governmental institutions like Amtrak and the US Export-Import Bank. 

\paragraph{Variation in Responses to Identical FOIA Request} Responses to our FOIA requests varied significantly. Most agencies offered logs of congressional correspondence, which record a date, sender, summary of the request, and other information used by agency staff to process and respond to requests. Logs generally include any written requests, as well as many phone and email records. For example, between May 2015 and December 2017, the Department of Justice Office of Administrative Law Judges received 132 emails, 109 telephone calls, and only 54 letters. Between 2007 and 2017, the Postal Regulatory Commission received 100 emails, 30 faxes, 173 letters, 118 calls. In this paper, we use ``contacts'' and ``letters'' interchangeably to refer to all modes of correspondence. 

Small agencies and regional offices had staff search their email history or provided hand-written records, which we then transcribed. Department Secretary offices generally queried a correspondence tracking database designed to track all correspondence. Still, our FOIA requests to sub-departmental components almost always recovered additional congressional correspondence records missing from central databases. As one central office FOIA officer put it, ``Legislative Affairs is supposed to be the front door for the department, but if somebody knows somebody, well...'' (personal communication, February 21, 2018). Because of such idiosyncratic relationships, capturing correspondence patterns that ``go around'' a Department Secretary's office is key to avoiding erroneous inferences about legislator behavior. For example, when chairs of the Homeland Security committee wrote about immigration enforcement issues, they almost always contacted the Department of Homeland Security (DHS) office of the Executive Secretary, but, at the same time, the Immigration Customs Enforcement (ICE) component of DHS directly received thousands of requests from a different set of legislators. Our systematic data collection ensures that we capture the totality of legislators' behavior.

\paragraph{Data processing and coding} Upon receiving records of congressional requests, we extracted names matching variations of legislators' names. We then merged in data about their position in Washington, district, and career, including ideology scores \citep{dwnominate2018}, committee membership \citep{StewartWoon2017}, and committee oversight \citep{LewisSelin2012}. We also made a considerable effort to verify and update committee membership data. 
In the online Appendix, we provide our procedure and replication code for converting the raw records from federal agencies into the data set required for our analysis. 

For a sample of \input{../tables/n-coded}requests, we use the text or summaries of letters to classify legislators' reasons for contacting federal agencies. Our coding process began with the authors coding a representative sample of records using our codebook (Appendix \ref{a:codebook}). We then trained Research Assistants. The first several thousand letters were double-coded. For example, of over 10,000 letters for the Environmental Protection Agency, the first 2,500 were double-coded. Our overall inter-coder agreement was 0.78, which rose to 0.9 when we subsetted our analysis to coding decisions where the coders had a great deal of certainty. We also developed subagency-specific coding rules throughout the hand-coding process where certain regular expressions indicated certain types of requests. For example, where documents containing the word ``rulemaking" consistently indicated that a legislator's request involved an agency's proposed rule, we assigned all observations containing the word ``rulemaking" yet uncoded by hand to the ``Policy-Rulemaking" category.\footnote{Hundreds of scripts for processing the raw data from each agency and applying any inductively-generated regular-expression-based coding are available on our GitHub, along with each script's full revision history and all written communication with RAs about processing and coding these data.}

We classified legislator requests into five categories: ``Individual Constituent Service'' (i.e., casework or advocacy on behalf of a group such as employees of a company), ``Nonprofit or Local Government Constituent Service'' (e.g., help with a grant application), ``Corporate Constituent Service'' (e.g., help with a specific government contract), ``Corporate Policy'' (policy work explicitly aimed to benefit a specific industry, like tariffs and subsidies), and ``Policy'' (general policy work related to legislation, budgets, or rulemaking). We define constituents broadly such that they need not be in a member's district. For example, Representative Tauscher of Wisconsin wrote to the Defense Commissary Agency on behalf of the Jelly Belly Candy Co., based in California. Jelly Belly was then ``given a chance to resolve issues" with their contract. We coded this case as ``Corporate Constituent Service,'' part of our broader measure of constituent service. We also consider constituent service as broader than individual casework. For example, we coded Senator Rubio asking the IRS for special treatment for residents of hurricane-affected parts of Florida as ``Individual Constituent Service.'' We note these ``hard cases'' to illustrate the boundaries of our coding scheme. Most contacts were easily parsed into either individual casework or policy work related to hearings, regulations, and legislation.


\subsection{Who Contacts the Bureaucracy and Why?} \label{s:descriptive} 
Before testing theories of how legislators' efforts to provide constituency service change as they acquire experience and power, we first use our extensive new data set to answer outstanding descriptive questions about representation in American politics. These descriptive findings regarding the level, variation, and reasons for legislator requests to federal agencies are only possible with our census of legislator requests. Overall, we find massive variation across legislators in their level of contact with federal agencies. We also find surprising consistency in the purpose of the communication; when legislators contact federal agencies, it is almost always to provide constituency service; only a small fraction focused on policy work. Further, we find that legislators are responsive to their constituency's demographic characteristics, but there is still significant variation in levels of service from quite similar districts.  

\subsubsection{Legislator's Contact with Federal Agencies Focus on Constituency Service}
Overall, when legislators contact federal agencies, they are helping constituents navigate the federal bureaucracy. Figure \ref{f:type2} shows the proportion of contacts for each of the five types of legislator requests in our hand-coded sample described above. The left-hand bar shows that 65\% of all contacts with federal agencies are made on behalf of individual constituents. Constituent service requests on behalf of individual corporations are a smaller percentage, 6\%, and 10\% of requests are on behalf of nonprofits and local governments. General policy work and policy work on behalf of specific industries account for less than 20\% of all requests made to federal agencies.  


\begin{figure}[hbt!]
\centering
\caption{Legislator Requests to Federal Agencies by Type 2007-2018} \label{f:type2}
\includegraphics[width = .8\textwidth]{figs/data_by_type-tall-1}
\end{figure}



To assess whether legislators shift their priorities as they gain experience and power, we further group requests into a broader ``constituency service"  category (including service for individuals, corporations, and nonprofits) and a broader ``policy work" category (including both general and industry-focused policy work). These two categories match those we go on to test in section \ref{s:prestige}. Using this broader classification, we find 74\% of contacts that chairs of committees make with federal agencies are focused on constituency service (compared to 82\% for non-chairs), and 78\% of contacts from members of prestige committees are focused on constituent service (compared to 81\% for members of non-prestige committees). Descriptively, constituency service is the main reason for legislators' requests regardless of their position in Congress or tenure in office. However, legislators in more powerful positions do appear to increase the ratio of policy work to constituency service.\footnote{We say that a House member is on a prestige committee if they are on Appropriations, Ways and Means, Rules, Budget, or Armed Services and if a senator is on Rules, Foreign Relations, Commerce, Budget, Armed Services, or Appropriations.} %CITE? 


\subsubsection{Levels of Contact with Federal Agencies are Highly Unequal}
Legislators vary significantly in how often they contact federal agencies. Gini coefficients for the number of contacts per year for the House and Senate are similar to those for income inequality in Mexico and the United States, respectively. Figure \ref{f:contact1} shows the average number of contact rates per year for House members (left-hand panel) and senators (right-hand panel). In the Senate, Robert Byrd (D-WV) averaged 510 contacts per year. Other senators---such as Charles Schumer (D-NY) and John McCain (R-AZ)---have similarly high levels of contact with federal agencies. But other senators contact at a much lower rate. On average, senators in our data contacted agencies 156 times per year.   
 
\begin{figure}
\centering
\caption{Variation in Average Legislator Requests by Percentile} \label{f:contact1} 
\begin{minipage}{\textwidth}
\includegraphics[width = \textwidth]{figs/percentiles-1}
\footnotetext{This figure presents the average number of contacts with federal agencies per year for House members (left-hand panel) and senators (right-hand panel), where the legislators' counts are sorted by their per year percentile rank. This reveals that senators and House members regularly contact federal agencies, but there is considerable variation in the level of contact across legislators.}
\end{minipage}
\end{figure}


We see a similar level of variation in the House, but with lower overall levels of contact with federal agencies, reflecting lower resources and fewer constituents than senators. Frank Wolf (R-VA) averaged 377 contacts per year. Like the Senate, other members of the House contacted at a much lower rate. For example, in her first year in Congress,  Michele Bachmann wrote only six letters in our data but would average 31 letters per year by the end of her time in Congress. Overall, House members averaged 52 contacts with federal agencies per year. But like the Senate, we large variability in the levels of contact across House members.  

Figure \ref{f:peryear} shows the number of requests per legislator over time, highlighting three Senators at the upper, middle, and lower parts of the distribution.

\begin{figure}
\centering
\caption{Variation in Legislator Requests by Year 2007-2017} \label{f:peryear} 
\begin{minipage}{\textwidth}
\includegraphics[width = \textwidth]{figs/counts-per-year-1}
\footnotetext{This figure presents the number of contacts with federal agencies per year for House members (left-hand panel) and senators (right-hand panel) over time. This reveals that senators and House considerable variation in the level of contact both within and across legislators.}
\end{minipage}
\end{figure}

\section{The Effect of Experience and Institutional Power}\label{s:results} 

Using this data set of requests to federal agencies, we now assess how legislators' changing position in Congress affects their provision of constituency service. First, we test theories rooted in legislator capacity by modeling the effects of institutional power on the overall \textit{level} of constituency service. Next, we test theories rooted in legislators' priorities by modeling the effects of institutional power on the overall \textit{ratio} of policy work to constituency service. Finally, we estimate the combined substantive impact of increasing capacity and shifting priorities taken together. 

\subsection{The Effects of Experience and Institutional Power on Levels of Constituency Service}\label{s:prestige}

 Our primary models are a series of difference-in-differences regressions, similar to the specifications in \cite{BerryFowler2016}. Our most stringent specifications examine changes that are within legislator and agency pairs.\footnote{We drop five member-congress level observations for congresses where the member switched political parties.} Specifically, we estimate regressions of the form: 

\begin{eqnarray}
Y_{ijt} & = & \boldsymbol{\beta}^{'} \textbf{Committee Position}_{it}  + \sum_{s = 1}^{6} \eta_{s} \text{I}\left(\text{tenure}_{it} = s\right) + \gamma_{ij} + \delta_{jt} + m_{it} + p_{it} + \epsilon_{ijt} \label{e:diff1}
\end{eqnarray}

Where $Y_{ijt}$ represents the number of requests legislator $i$ makes to agency $j$ in year $t$. Our analysis in this section thus focuses on the legislator-agency-year level. $\gamma_{ij}$ is a fixed effect for the legislator-agency pair. This fixed effect accounts for legislators' characteristics, such as legislators who are more skillful at filling constituency service requests than other legislators. Critically for our research design, this fixed effect also enables us to account for time-invariant constituent demand for constituency service with an agency, ensuring differences in constituent demand do not drive our results. It also accounts for characteristics such as the population of a state, its residents' demographic characteristics, and local industries that might be particularly likely to request help with specific agencies. This difference-in-difference design ensures that coefficients $\boldsymbol{\beta}$ capture variation related to changes in institutional power or experience, not other factors that may vary across districts, legislators, or agencies. The model also accounts for the different periods for which data were available from each agency. $\delta_{jt}$ is an agency-year fixed effect. This takes into account agency-level common shocks across legislators in the provision of constituency service. 

Assuming that legislators' trends in the level of requests follow parallel paths, $\boldsymbol{\beta}$ represents the average effect of changing institutional power on a legislator's provision on constituency service. We focus on three measures of a legislator's committee position: (1) whether they are a committee chair, (2) whether they are a ranking member of a committee, and (3) whether they are members of a prestige committee. We focus on these committee positions because each represents a different way legislators can acquire more power while in Congress. As a legislator becomes a committee chair or ranking member, they have increased responsibilities when drafting and revising legislation. They also have increased access to committee resources to accomplish policy goals, particularly the power to direct committee staff. Similarly, legislators who join more prestigious committees gain the opportunity to shepherd policy through the legislative process.

As \cite{BerryFowler2016} note, changes in legislators' committee assignments are often due to circumstances outside of the legislator's control, such as changing majority status, retirements on a committee, or exclusion due to losses from a previous election \citep{GrimmerPowell2013}. To violate the parallel trends assumptions, it would need to be the case that legislators differentially altered their rates of constituency service in anticipation of joining particular committees. To help avoid this violation, we include a series of controls that capture time-varying characteristics of a legislator that might confound our inference about the effect of committee prestige. Because legislators may make more requests to a president of the same party \citep{BerryBurdenHowell09}, it is a particular concern that legislators obtain new committee assignments when their party moves into or out of the majority or at the same time as the president party changes. To address these concerns, we include an indicator for whether the legislator's party is the majority in year $t$, $m_{it}$ and if the legislator of Congress is from the same party as the president in year $t$, $p_{it}$. Throughout, we cluster our standard errors at the legislator level.

In this same regression we also include indicators for legislators' first six years in Congress, $ \sum_{s = 1}^{6} \eta_{s} \text{tenure}_{it}$. The effects of interest $\eta_{1}, \eta_{2}, \hdots, \eta_{6}$ describe how a legislator's provision of constituency service at levels of seniority between one and six years differ from legislators who serve beyond six years. We focus on constituency service levels in each of the first six years of a legislator's tenure in Congress to assess how constituency service changes over their initial years in Congress. This design allows us to assess the extent to which new legislators face start-up costs. This specification, however, is ill-equipped to assess how electing a new representative affects the amount of constituency service a district receives. We address this in Section \ref{s:tenure_dist} by examining how electing a new representative affects the number of constituency service requests made on behalf of a district with a different difference-in-differences specification at the district-year level.   

%\subsection{The Effects of Experience and Institutional Power on the Level of Constituency Service} 


\begin{table}[hbt!]
\caption{The Effect of Experience and Institutional Power on Constituency Service} \label{t:models_total}
\begin{minipage}{\textwidth}
\begin{center}
% \input{../tables/FinalTable_1} % stata table was missing FE
\begin{tabular}[t]{lcccc}
\toprule
\textbf{ } & \textbf{(1)} & \textbf{(2)} & \textbf{(3)} & \textbf{(4)}\\
\midrule
\textbf{Dependent Variable} & \textbf{Count} & \textbf{Count} & \textbf{Count} & \textbf{Log(Count+1)}\\
\midrule
Committee Chair & \num{0.715} & \num{0.271} & \num{0.275} & \num{0.049}\\
 & (\num{0.151}) & (\num{0.090}) & (\num{0.090}) & (\num{0.012})\\
Ranking Member & \num{0.842} & \num{0.153} & \num{0.170} & \num{0.031}\\
 & (\num{0.154}) & (\num{0.094}) & (\num{0.094}) & (\num{0.010})\\
Prestige Committee & \num{0.469} & \num{0.100} & \num{0.093} & \num{0.026}\\
 & (\num{0.067}) & (\num{0.051}) & (\num{0.052}) & (\num{0.010})\\
First Year & \num{-0.301} & \num{-0.512} & \num{-0.494} & \num{-0.103}\\
 & (\num{0.053}) & (\num{0.075}) & (\num{0.073}) & (\num{0.012})\\
Second Year & \num{-0.067} & \num{-0.275} & \num{-0.291} & \num{-0.042}\\
 & (\num{0.060}) & (\num{0.072}) & (\num{0.072}) & (\num{0.011})\\
Third Year & \num{-0.046} & \num{-0.189} & \num{-0.208} & \num{-0.030}\\
 & (\num{0.063}) & (\num{0.061}) & (\num{0.060}) & (\num{0.009})\\
Fourth Year & \num{0.026} & \num{-0.135} & \num{-0.158} & \num{-0.018}\\
 & (\num{0.067}) & (\num{0.060}) & (\num{0.057}) & (\num{0.009})\\
Fifth Year & \num{-0.046} & \num{-0.135} & \num{-0.139} & \num{-0.024}\\
 & (\num{0.059}) & (\num{0.043}) & (\num{0.042}) & (\num{0.007})\\
Sixth Year & \num{0.049} & \num{-0.029} & \num{-0.011} & \num{-0.014}\\
 & (\num{0.073}) & (\num{0.056}) & (\num{0.054}) & (\num{0.007})\\
Majority & \checkmark & \checkmark & \checkmark & \checkmark\\
\midrule
President's Party & \checkmark & \checkmark & \checkmark & \checkmark\\
All Legislators & \checkmark & \checkmark &  & \checkmark\\
Served At Least 2nd Term &  &  & \checkmark & \\
Observations & \num{412111} & \num{412111} & \num{388997} & \num{412111}\\
Year x Agency FE & \checkmark & \checkmark & \checkmark & \checkmark\\
Legislator x Agency FE &  & \checkmark & \checkmark & \checkmark\\
\bottomrule
\multicolumn{5}{l}{\rule{0pt}{1em}\footnotesize Robust standard errors in parentheses, clustered by legislator.}\\
\end{tabular}
 % this one is from replication.rmd
\end{center}
\footnotetext{This table shows how the number of contacts changes as legislators acquire more expiernece and power in Congress. Column 1 shows the average differences across committee assignments and years in Congress. Column 2 presents the difference-in-differences estimates. Column 3 subsets to legislators who serve at least 3 years in Congress. Column 4 takes the Log of the counts + 1 as the dependent variable.}
\end{minipage}
\end{table}

Table \ref{t:models_total} provides the coefficient estimates from Equation \ref{e:diff1}. We focus first on the estimated effect of increased committee prestige. Table \ref{t:models_total} shows that as legislators acquire more prestige, their rates of constituency service increase. All coefficients represent the average additional requests per year \textit{per agency}; per legislator per year effects are simply these coefficients times 83 (the number of agencies).
% IN THE RAW DATA, THERE ARE 90 AGENCIES, BUT MAYBE SOME GET DROPPED? 
Model 1 (the first column of Table \ref{t:models_total}) shows that this is true in a cross-sectional comparison across legislators. Model 1 excludes the legislator-agency and year-agency fixed effects, but it does include controls for majority status and being from the same party as the president. 

\subsubsection{The Effect of Institutional Power on Levels of Constituency Service}\label{s:prestigeresults} 

Table 1 shows that committee chairs, ranking members, members of prestige committees, and oversight committee members provide substantially more constituency service than other legislators. However, these cross-sectional differences conflate a legislator's institutional position with other legislator characteristics. If legislators who are better at their jobs or exert more effort are also selected for more prestigious committee positions, then the estimates from Model 1 confound legislators' overall ability with their institutional position.   


To address potential confounding in across-legislator comparisons, the estimates from Model 2 (Column 2 of Table \ref{t:models_total}) provides the estimated effects from the difference-in-differences specification in Equation \ref{e:diff1}. Across all measures of institutional power, we find that more power increases the number of requests that legislators make. Consider first the effect of being a committee chair. We estimate that becoming a committee chair causes an increase of 0.27 requests \textit{per agency} (95-percent confidence interval [0.09, 0.45]). Across all 90 agencies, this represents an increase of approximately 24 additional requests per year, 24.1\% of the average number of requests per year in our data. There is a smaller increase for individuals who become ranking members and those who join a Prestige Committee, though the increase is statistically significant for the prestige committee. Becoming a ranking member of a committee causes an increase of 0.15 contacts per agency while joining a prestige committee causes a 0.27 per agency increase in the number of contacts a member of Congress makes.

We estimate that the experience gained between the first and second year in Congress causes an increase of 0.24 requests \textit{per agency}. The experience gained between the first and seventh years causes an increase of 0.51 per agency. Across all 90 agencies, this represents an increase of approximately 46 additional requests per year, 45.5\% of the average number of requests per year in our data. There is a smaller increase after the second year. The experience gained between the second and seventh year causes an increase of 0.28 per agency, an increase of approximately 25 additional requests per year, 45.5\% of the average number of requests per year in our data.



\begin{figure}[hbt!]
\centering
\caption{Predicted Number of Total Letters (Within Legislator Difference in Differences) 2007-2018} \label{f:m-total-predicted}
\includegraphics[width = .8\textwidth]{figs/m-total-predicted-4}
\end{figure}

Figure \ref{f:m-total-predicted} shows the predicted total number of letters by year in Congress and committee chair status (comparing predictions for counterfactuals where the same legislator did and did not receive a chairmanship in their sixth year in Congress).\footnote{Predictions are based on a legislator-agency pair where (1) the legislators' average annual contacts equaled the overall average, (2) the legislators' number of contacts with the agency equal the average received by that agency, (3) and the agency received an average number of letters.} In their first year, legislators make significantly fewer requests to agencies than they do in the following year. Subsequent increases are less significant. However, there is a significant difference between the same legislator as a committee chair and not.

The findings in Table \ref{t:models_total} are robust to alternative specifications and measures of the dependent variable. For example, we might be concerned that legislators with exceptionally high levels drive the results. The fourth column shows that we obtain the same findings if we use $\log (Y_{ijt} + 1)$ in our difference-in-differences specification. Further, our results are not due to differential attrition. The third column shows that we obtain nearly identical results if we restrict our analysis to legislators who serve beyond three years.    

This section has shown that acquiring power in Washington causes legislators to increase constituency service levels. This increase occurs across all three measures of committee position that we examine but is most robust for committee chairs. 


\subsubsection{The Effect of Legislator Experience on Levels of Constituency Service}\label{s:tenure} 

As legislators acquire more power in Washington, they increase their provision of constituency service. While this suggests that more powerful legislators are paying more attention to their constituents, it could still be the case that as legislators gain experience in Washington, they decrease their constituency service provision. To test whether this is the case, we use the estimates in Table \ref{t:models_total} but now focus on the coefficients in legislators' first six years in office. The reference group is representatives who have served longer than six years.\footnote{Interpreting these coefficients requires that we assume the effects of tenure and committee assignment are linearly separable. This assumption is reasonable because most legislators do not become chairs, ranking members, join prestige committees in their first six years in Congress, almost none in their first two years.} 

The first column of Table \ref{t:models_total} shows that there are large cross-sectional differences: legislators in their first year make fewer contacts than more experienced legislators. First-year legislators make approximately 0.255 fewer requests per agency than legislators in their seventh year or beyond. This difference shrinks in the second year and then is mostly gone. But we advise caution in interpreting the differences in Column 1 because they conflate the effect of increased experience in Congress with other characteristics that may correlate with whether a legislator remains in office and thus whether we observe them in later years.   

To account for possible differences in legislators who obtain different levels of tenure, the second column of Table \ref{t:models_total} estimates the difference-in-differences specification in Equation \ref{e:diff1}. The tenure coefficients show that legislators provide less constituency service in their first year in office. As they acquire experience in Congress, they make more requests to federal agencies. In their first year in office, legislators provide 21.33 (90 $\times$ (.512 - .275)) fewer requests per agency than legislators in their second year and 29.07 (90 $\times$ (.512 - .189)) fewer requests than legislators in their third year---both differences are statistically significant at conventional levels. The overall increase in levels of constituency service from a legislator's first to third year is similar in size to the increase that comes from becoming a member of the oversight committee. Once legislators enter their fourth year, their behavior no longer differs from more experienced legislators. We find small and statistically insignificant differences for legislators in their fourth through sixth years. As legislators acquire experience and build their office's organizational capacity in their first two years, they make more contacts with federal agencies.  

As with the analysis of committee prestige, the findings in Table \ref{t:models_total} are robust to alternative specifications. Despite the difference-in-difference design, we might still be concerned that the set of legislators who serve in their third year are different than the legislators who serve in the first years. If this were the case, then our findings would be the result of both the experience and a selection effect due to House members who win reelection, a potential indication that they are better able to perform the job than other legislators. To address the potentially different samples in each year, the third column of Table \ref{t:models_total} assesses the changes in the number of contacts of federal agencies for legislators who serve in Congress for at least three years. The pattern is similar: legislators initially provide less constituency service in their first two years than they do in subsequent years. And Column 4 in Table \ref{t:models_total} shows that the results are robust to analyzing $\log(Y_{ijt} + 1)$, ensuring that our results are not because of outliers. Additional models in Appendix \ref{s:appendix_models} estimate the same models as Table \ref{t:models_total} on hand-coded subsets of the data, showing similar results. 

\subsubsection{The Effect of Electing a New Representative on the Level of Constituency Service}\label{s:tenure_dist}

In the previous section, we showed that legislators make fewer requests to federal agencies in their first year in office but that the number of requests stabilizes after their third year. We now turn to a related question: how does the level of constituency service to a district change after the election of a new representative? Rather than examining changes in the number of requests by making within-legislator comparisons, we now make within-district comparisons to assess how electing a new legislator affects the total number of requests a district's representative makes. In other words, within-district comparisons enable us to assess the costs or benefits of electing a new representative compared to an incumbent.%\footnote{For an illustration of changes in one district over time, see Appendix \ref{s:wi}, which shows the number of requests from legislators representing Wisconsin's 7th District over time}. %% FOOTNOTE IF WE MOVE EXAMPLE BACK TO APPENDIX

%\subsection{Example} \label{s:wi}

To illustrate our findings regarding the effect of legislator experience on contacts with the federal bureaucracy, Figure \ref{f:wi} shows the change in contacts from legislators representing Wisconsin's 7th district in the House (top) and the Senate (bottom). 
In line with our overall findings from our cross-sectional and difference-in-difference designs described below, newly-elected Representative Sean Duffy initially provided less constituency service than twenty-term Representative Dave Obey but was on par with Obey's average number of contacts by year three. Indeed, the only year in our data that there were fewer contacts from the representative of Wisconsin's 7th district than the average member of the House (the dotted line in the top panel of \ref{f:wi}) was Representative Duffy's first year in Congress. Figure \ref{f:wi} shows similar dips in the level of constituency service in the transition from Senator Feingold to Senator Johnson and from Senator Kohl to Senator Baldwin. The only years in which Wisconsin's senators contacted the federal bureaucracy fewer times than the Senate average (the dotted line in the bottom panel of \ref{f:wi}, were Baldwin's first year and Johson's first five years in the Senate. 

\begin{figure}[hb!]
\centering
\caption{Example: The Effect of Electing New Legislators in Wisconsin} \label{f:wi}
\includegraphics[width = .8\textwidth]{figs/districts/(WI-07)}
\includegraphics[width = .8\textwidth]{figs/districts/(WI)}
\end{figure}

To make this type of district-level comparison systematically, we change the level of our analysis from the legislator to the district and focus now on the number of contacts made from the representative of a particular state or district $i$ in a year $t$, $Y_{it}$. We again use a difference-in-differences approach to account for district-specific characteristics and over-time changes in how legislators provide constituency service. Specifically, we estimate regressions of the form: 


\begin{eqnarray}
Y_{it} & = & \beta_{1}\text{New Member}_{it} + \sum_{s = 2}^{6} \beta_{s} \text{tenure}_{s[it]} + \gamma_{i} + \delta_{t} + \epsilon_{it} \label{e:district1} 
\end{eqnarray}

Where $\gamma_{i}$ is a district-specific fixed effect that accounts for each district's particular demographic characteristics, along with the levels of demand from district residents. $\delta_{t}$ is a year fixed effect that takes into account common shocks. Our key result of interest, $\beta_{1}$, is the effect of a district electing a new representative. To understand how the effect of a new representative changes over time, we estimate district-level differences for a legislator's second ($\beta_{2}$) through sixth-year ($\beta_{6})$.\footnote{It is worth noting that this treatment is fundamentally different for a district than within-legislator variation. In each election, each district either allows its incumbent to acquire another term or replaces her. This is different from within-legislator comparisons because legislators can only acquire more tenure or leave the chamber. A within-legislator analysis estimates the service provided by incumbents with more or less experience; it cannot estimate the impact of the choice of an incumbent or a new representative incumbent.} % TODO THIS FOOTNOTE IS REPETITIVE.

The first column of Table \ref{t:district2} provides a simple difference-in-means for districts represented by a new member and then for legislators in their first six years in office. Comparing across districts, districts represented by new legislators receive substantially lower levels of constituency service. On average, districts with a new representative have 35.2 fewer constituency service requests made on their behalf. The magnitude of this difference shrinks for districts represented by legislators in their second year (23.75 fewer constituency service requests). It then reaches a relatively stable number for districts represented by legislators in their third through sixth years.%, with some slight evidence that legislators make more contacts in election years---the even-numbered years of tenure for the vast majority of legislators in our sample. 



\begin{table}[hbt!]
\caption{The Effect of Electing New Members on a District's Level of Constituency Service} \label{t:district2}
\begin{minipage}{\textwidth}
\begin{center}
\begin{tabular}{l*{4}{c}}
\toprule
                    &\multicolumn{1}{c}{(1)}&\multicolumn{1}{c}{(2)}&\multicolumn{1}{c}{(3)}&\multicolumn{1}{c}{(4)}\\
\midrule
\textbf{Dependent Variable} & \textbf{Count} & \textbf{Count} & \textbf{Count} & \textbf{Count}\\
\midrule
New Legislator      &      -35.23&      -35.55&      -14.89&      -123.5\\
                    &     (4.445)&     (4.500)&     (2.627)&     (13.84)\\
Legislator 2nd Year &      -23.75&      -20.31&      -4.402&      -79.99\\
                    &     (4.464)&     (3.949)&     (2.662)&     (11.34)\\
Legislator 3rd Year &      -13.08&      -13.53&      -1.630&      -49.48\\
                    &     (4.886)&     (4.448)&     (2.586)&     (16.07)\\
Legislator 4th Year &      -12.43&      -9.077&       0.268&      -26.92\\
                    &     (5.216)&     (4.276)&     (2.736)&     (16.30)\\
Legislator 5th Year &      -14.92&      -11.58&      -3.810&      -31.58\\
                    &     (4.416)&     (3.591)&     (2.128)&     (13.11)\\
Legislator 6th Year &      -13.56&      -5.216&      -1.638&      -2.500\\
                    &     (5.104)&     (3.790)&     (2.239)&     (14.46)\\
\midrule
District Fixed Effects&            &  \checkmark&  \checkmark&  \checkmark\\
Year Fixed Effects  &            &  \checkmark&  \checkmark&  \checkmark\\
All Districts       &  \checkmark&  \checkmark&            &            \\
House Only          &            &            &  \checkmark&            \\
Senate Only         &            &            &            &  \checkmark\\
Observations        &        6578&        6578&        5338&        1240\\
\bottomrule
\multicolumn{5}{l}{\footnotesize Robust standard errors in parentheses, clustered at district level}\\
\end{tabular}

\end{center}
\footnotetext{This table shows how constituent service at the district level changes over time. Model 1 is a cross sectional comparison excluding district and year fixed effects. The second column is a district x year difference in differences model. Column 3 focuses the diff-in-diff on legislators who survive their first election.}
\end{minipage}
\end{table}


To account for differences in district size, demographics, and demand for constituency service, the second column of Table \ref{t:district2} estimates the difference-in-differences from Equation \ref{e:district1}. In this specification, we see a large causal effect of a new member taking over: electing a new member causes a decrease of 36 constituency service requests (95-percent confidence interval [-44, -27]), a sizable change in the number of service requests representatives make on behalf of their new constituents. %TODO PERCENT CHANGE 
The effect of electing a new representative, however, dissipates quickly. Districts represented by a legislator in her second year of service receive 12 fewer constituency service requests---still a substantively significant decrease in contact with federal agencies, but not as drastic as the first year decrease. After the second year, the differences are smaller in magnitude. This phenomenon--new legislators providing substantially fewer requests--persists when examining the House (Column 3) and the Senate (Column 4) separately. In short, new legislators make fewer contacts for their constituents than well-established elected officials.  


\paragraph{The Costs of Newly Elected Members} Taken together, our results demonstrate that new legislators provide much less constituency service. Legislators in their first year provide much less constituency service than they do in their second year and reach a stable level of service in their third year. Further, when districts elect a new representative or senator, they experience a sharp decrease in constituency service requests made on their behalf. Rather than experienced legislators forgetting about their districts, our evidence suggests that newly elected legislators experience substantial start-up costs and struggle to provide the levels of service that experienced legislators deliver to their constituents.  


\subsection{The Effect of Experience and Institutional Power on Legislators' Priorities}\label{s:priority} 

To assess legislators' ratio of policy work to constituency service, we use the hand-coded data described in Section \ref{s:data}. The dependent variable in Table \ref{t:models_ratio} is the number of policy requests divided by the number of constituency service requests per legislator per year. These models test whether legislators' priorities shift among goals as they gain experience and power in Congress.

\begin{table}
\begin{center}
\begin{minipage}{\textwidth}
\caption{The Effect of Expierence and Institutional Power on the Ratio of Policy Work to Constituency Service} \label{t:models_ratio}
\centering
% \begin{tabular}{l*{2}{c}}
\toprule
                    &\multicolumn{1}{c}{(1)}&\multicolumn{1}{c}{(2)}\\
\midrule
Prestige            &      0.0220&     -0.0105\\
                    &   (0.00777)&    (0.0101)\\
Chair               &     -0.0744&     -0.0875\\
                    &    (0.0157)&    (0.0175)\\
Ranking Minority    &    0.000387&     -0.0346\\
                    &    (0.0138)&    (0.0142)\\
First Year          &      0.0428&      0.0223\\
                    &   (0.00918)&    (0.0115)\\
Second Year         &      0.0744&      0.0546\\
                    &   (0.00888)&    (0.0111)\\
Third Year          &      0.0400&      0.0235\\
                    &   (0.00888)&    (0.0107)\\
Fourth Year         &      0.0357&      0.0192\\
                    &   (0.00970)&    (0.0108)\\
Fifth Year          &     0.00801&    -0.00254\\
                    &   (0.00994)&    (0.0105)\\
Sixth Year          &      0.0533&      0.0431\\
                    &   (0.00933)&   (0.00940)\\
\midrule
Majority            &            &  \checkmark\\
Legislator Fixed Effects&            &  \checkmark\\
Year Fixed Effects  &            &  \checkmark\\
Observations        &        6442&        6442\\
\bottomrule
\multicolumn{3}{l}{\footnotesize Robust standard errors in parentheses, clustered at legislator level}\\
\end{tabular}
 % stata version was missing FE
\begin{tabular}[t]{lcc}
\toprule
  & (1) & (2)\\
\midrule
\textbf{Dependent Variable} & \textbf{Ratio} & \textbf{Ratio}\\
\midrule
Committee Chair & \num{-0.070} & \num{-0.071}\\
 & (\num{0.016}) & (\num{0.017})\\
Ranking Member & \num{-0.002} & \num{-0.029}\\
 & (\num{0.013}) & (\num{0.014})\\
Prestige Committee & \num{0.022} & \num{-0.004}\\
 & (\num{0.008}) & (\num{0.009})\\
First Year & \num{0.057} & \num{0.017}\\
 & (\num{0.009}) & (\num{0.014})\\
Second Year & \num{0.064} & \num{0.024}\\
 & (\num{0.008}) & (\num{0.013})\\
Third Year & \num{0.059} & \num{0.021}\\
 & (\num{0.009}) & \vphantom{1} (\num{0.012})\\
Fourth Year & \num{0.035} & \num{-0.003}\\
 & (\num{0.009}) & (\num{0.012})\\
Fifth Year & \num{0.027} & \num{-0.003}\\
 & (\num{0.010}) & (\num{0.011})\\
Sixth Year & \num{0.041} & \num{0.012}\\
 & (\num{0.009}) & (\num{0.010})\\
\midrule
Majoirty & \checkmark & \checkmark\\
President's Party & \checkmark & \checkmark\\
Observations & \num{6442} & \num{6442}\\
Year Fixed Effects & \checkmark & \checkmark\\
Legislator Fixed Effects &  & \checkmark\\
\bottomrule
\multicolumn{3}{l}{\rule{0pt}{1em}\footnotesize Robust standard errors in parentheses, clustered by legislator.}\\
\end{tabular}
 % this one is from replication.rmd
\footnotetext{This table shows how the proportion of contacts focused on constituency service changes as legislators acquire more expiernece and power in Congress. Column 1 shows average differences across committee assignments and years in Congress. Column 2 presents difference-in-differences estimates.}
\end{minipage}
\end{center}
\end{table}






Table \ref{t:models_ratio} shows that legislators increase the ratio of policy work to constituency service as they obtain more experience and prestigious committee assignments in Congress. The first column of Table \ref{t:models_ratio} shows how the proportion of policy work to constituency service differs \textit{across} legislators' in their first six years in office and for legislators who acquire committee positions. 

While the ratio of policy work to constituency service is conditional on the levels of each, the inference we wish to make about the ratio does not depend on these levels; we are not using the ratio to infer the level (e.g., that a lower share of constituency service means a lower level). Instead, the theory of prioritization is directly about the ratio, regardless of the level. Levels may interact with the ratio, but not in ways that do not mean the same thing for our theory: that legislators are prioritizing one thing over the other. 

Column 2 of Table \ref{t:models_ratio} provides the estimated effects from the difference-in-differences specification. While there is little evidence that time in Congress affects legislator priorities, institutional positions do. We estimate that becoming a committee chair causes the ratio of constituency service to policy work to decrease by 0.07 (95-percent confidence interval [-0.04, -0.10 ]). Becoming a ranking member of a committee causes a decrease of 0.03 in the ratio.


\begin{figure}[hbt!]
\centering
\caption{Predicted Number of Total Letters (Within Legislator Difference in Differences) 2007-2018} \label{f:m-ratio-predicted}
\includegraphics[width = .8\textwidth]{figs/m-ratio-predicted-4}
\end{figure}

Figure \ref{f:m-ratio-predicted} shows the predicted ratio of constituency service to policy work by year in Congress and committee chair status (comparing predictions for counterfactuals where the same legislator did and did not receive a chairship in their sixth year in Congress). There is little change in priorities as members gain experience. However, there is a significant difference between the same legislator as a committee chair and a counterfactual where they are not.



\section{The Effects of Demand for Constituency Service}\label{s:demand} 

This section shows that demand for constituency service affects the level of constituency service that members provide. However, it does not appear that demand-side shifts can explain the specific within-legislator or within-district variation we observe with changing committee assignments and increased tenure. First, we show that constituent demand does drive legislators' constituency service requests to the agencies best suited to address the district's needs. Then, we show that this demand does not shift within or across legislators in ways that we would expect to see if shifting constituent demand explained the within-legislator and within-district variation in constituency service discussed in section \ref{s:prestigeresults}.

\subsection{District Characteristics Affect the Provision of Constituency Service}

The characteristics of their districts help inform which agencies legislators contact. We find that population size correlates with the overall number of requests and that constituency characteristics--the proportion of veterans and the proportion over 65--correlate with the distribution of requests across agencies. These correlations provide face-validity for our measures of representation, but they also suggest that cross-sectional comparisons may conflate legislator choices with characteristics of districts. Given this potential conflation, our models below include fixed effects for each legislator-agency pair, leveraging within-district and within-agency variation.  

We expect senators who represent larger states to make more requests. Senators from larger states have a larger number of constituents to serve, and they receive a larger budget to handle that increase in requests.
Figure \ref{f:stateSize} shows that this is the case: senators from larger states provide more constituency service on average. Senators from larger states, like John Cornyn (R-TX), Barbara Boxer (D-CA), and Pat Toomey (R-PA), average more requests per year than legislators from smaller states. While the number of legislator requests is associated with population size, Figure \ref{f:stateSize} also shows significant variation in the level of service that senators provide, even among states of similar sizes.  

\begin{figure}
\centering
\caption{Average Number of Requests per Senator per Year 2007-2018 by State Population.} \label{f:stateSize}
\includegraphics[width = \textwidth]{figs/pop-1}
\end{figure}

We expect the number of times a legislator contacts a particular agency to correlate with their districts' demographic composition. To assess the correlation between demographic characteristics and the rates legislators contact agencies, we focus on two example agencies: the Veterans Administration (VA) and the Social Security Administration (SSA). We measured the prevalence of two groups in the district: veterans---the residents who might plausibly need assistance navigating the VA---and residents who are over 65 years of age and therefore satisfy the age eligibility for social security. We then run a simple bivariate regression of the total number of contacts a legislator made to each agency on the proportion of constituents who are veterans or who are over 65.  
In both instances, we find a correlation between district composition and the number of times legislators contact the agency. 

%% MOVE THIS TO APPENDIX AND ADD REGRESSION TABLE
%For example, for the VA, we find that a one percentage point increase in the proportion of residents who are veterans is associated with an increase of about .84 constituency service requests to the agency. Similarly, a one percentage point increase in the proportion of residents over 65 is associated with an increase of an additional 0.03 requests made to the agency. Overall, this suggests that legislators' efforts are correlated with the district's demographic composition.  

%% MORE NEEDED HERE

\subsection{Do Voters Demand More of More Powerful Legislators?}

Can variation in demand for constituency service explain why more experience and prestigious committee positions provide more constituency service? 

We limited the influence in demand when assessing how power and experience affect levels of constituency service above. For example, our empirical strategies in Sections \ref{s:prestige}, \ref{s:tenure}, and \ref{s:priority} account for static demand based on characteristics of the district. Districts composed of veterans might see more demand for assistance with the Veterans' Administration, or districts with older residents might have greater demand with the social security administration. Because our analyses include either legislator-agency or district fixed effects, we compare how the levels of constituency service change holding constant demand related to fixed district characteristics. Furthermore, theories of legislator capacity suggest that legislators use their increased capacity and resources to solicit constituency service requests and thus generate demand. Constituent demand driven by shifts in behavior is indeed a necessary part of the increased constituency service we attribute to increased capacity and resources in Sections \ref{s:prestige}.

Yet, we might expect that a legislator's experience or power in Congress could affect constituency demand, even without legislators using their increased capacity and resources to generate demand, as implied by theories focusing on legislator capacity. Constituents could, for example, direct more of their demands to legislators who are more powerful or who have been in Congress for longer because they don't know or trust new representatives (holding constant legislators' levels of soliciting constituency service requests). In this section, we investigate whether that additional constituent demand could plausibly explain our results. We find limited effects of legislator tenure on demand for constituency service. 

This section presents a more direct test of whether constituent demand explains variation in the level of constituency service that legislators provide. If constituents shift demand to more experienced legislators, and such a shift could explain levels of constituency service, then we should observe such shifts when a new member is elected. Suppose constituents shift demand based on legislator experience (as required for shifts in demand to explain our results). In that case, they should redirect their demands away from newly elected legislators towards other representatives. The most natural target for the constituent demands would be one of the senators representing the constituent's state.

To assess whether constituents redirect demand towards other more experienced legislators when new members replace their more experienced incumbent representative, we examine how experienced legislators' levels of constituency service change in response to having new representatives in their state. We measure new members in the state in two ways: either the proportion of House members and senators in the state who are new or an indicator for whether there is a new House member or senator in the state. As in Section \ref{s:prestigeresults}, we measure the number of requests made by a district's representative in a particular year. Using this dependent variable, we estimate a series of difference-in-differences regression where the treatment is new members in the state. We include district and year fixed effects. We restrict the regression to incumbent legislators only because we are interested in assessing whether constituents with new legislators direct their constituency service requests to these incumbents.   

Table \ref{t:spill1} presents the estimates of this regression. The first two columns are estimated on all incumbent legislators. They show that neither the proportion of new members nor a new member significantly affects the level of constituency service that other legislators in the state provide. Columns 3 and 4 of Table \ref{t:spill1} reveal the same pattern when focusing on senators only. The estimated effects of new legislators do not approach statistical significance.  

   

\begin{table}[hbt!]
\caption{Little Evidence of Spillovers from New Legislators} \label{t:spill1}

\begin{minipage}{\textwidth}
\begin{center}
\begin{tabular}{l*{4}{c}}
\toprule
                    &\multicolumn{1}{c}{(1)}&\multicolumn{1}{c}{(2)}&\multicolumn{1}{c}{(3)}&\multicolumn{1}{c}{(4)}\\
                    \midrule
\textbf{Dependent Variable} & \textbf{Count} & \textbf{Count} & \textbf{Count} & \textbf{Count} \\
\midrule
Proportion New Legislators&       5.143&            &      -1.494&            \\
                    &     (8.089)&            &     (20.06)&            \\
At Least One New Legislator&            &       1.625&            &       3.847\\
                    &            &     (2.031)&            &     (4.812)\\
\midrule
District Fixed Effects&  \checkmark&  \checkmark&  \checkmark&  \checkmark\\
Year Fixed Effects  &  \checkmark&  \checkmark&  \checkmark&  \checkmark\\
Senators Only       &            &            &  \checkmark&  \checkmark\\
Observations        &        6080&        6080&        1182&        1182\\
\bottomrule
\multicolumn{5}{l}{\footnotesize Robust standard errors in parentheses, clustered at district level}\\
\end{tabular}

\end{center}
\end{minipage}
\end{table}

\section{Discussion}

%% 1 Summary of findings

%% 2  Implications for theory (evidence that capacity matters, evidence that priorities change, the importance of studying both at the same time)

%% 3  Implications for policy (term limits, congressional staff, anything else)

%% 4 Future research (things one can do with these data; cite our other stuff, what else?) 


%% SUMMARY OF FINDINGS - MOVE TO CONCLUSION?
The vast majority of contracts with federal agencies focus on constituency service. While there is massive inequality in the quantity of service provided by different members, we show that this is not the result of long-serving members devoting less attention to their district over time as the ``Patomic Fever'' hypothesis and those advocating for term limits suggest. We do find evidence that legislators prioritize policy work as they acquire positions of institutional power. However, simultaneous increases in capacity that come with positions of institutional power more than offset shifting priorities. Put differently, we find evidence that shifting priorities and increasing capacity both affect legislator behavior. Critically, the magnitude of the effect of increased capacity is large enough that the district constituency receives no less particularistic service from long-serving and powerful legislators. 

%Using a robust research design that includes both cross-sectional and within-legislator variation, we show that...
In Section \ref{s:presitge} we showed that legislators provide more constituency service as they gain experience. In section \ref{s:priorty} we showed that legislators also increase their ratio of policy work to constituency service when they gain experience and power. These latter two findings have potentially countervailing effects on the overall level of constituency service (accounting for simultaneous shifts in capacity and priority). The key outcome is the net effect on levels of constituency service, accounting for the effects of both increasing capacity on the level of constituency service (Table \ref{t:models_total}) and shifting priorities on the ratio of policy work to constituency service (Table \ref{t:models_ratio}).


%% IMPLICATIONS FOR THEORY 
\subsection{Implications for Theory}

% SIMULTANEOUS EFFECTS 
Our finding that shifts in capacity and priorities simultaneously affect legislator behavior has implications for the study of legislator careers. Our measure of legislator contact to the bureaucracy is only one type of behavior that is likely affected by simultaneous changes in capacity and priority. Legislator capacity and priorities (and thus our theory) may apply to a wide variety of other behaviors, including. For example, the volume of legislative work, oversight reports, or hearings that result from a legislator's office depend both on their capacity to do that work and the relative priority they place on each task. Research designs that aim to identify the cause of any increased productivity as elected officials gain experience or achieve a position of institutional power must be able to tell whether the outcome was the result of shifting capacity, shifting priorities, or some combination of both. 

Furthermore, we find evidence for multiple mechanisms that contribute to the overall effect of increased capacity. We find that legislator experience matters. The volume of legislator requests to the bureaucracy increases dramatically in the first few years, well before legislators generally acquire committee chairships or other positions of institutional power that come with additional staff resources. This large, short-lived effect of being a new legislator is evidence for the Organizational Capacity mechanism we theorize. Because new legislators rarely become committee chairs, the large effect of gaining a committee chair position is evidence for the institutional power mechanism that we theorize. 

% CONSTITUENCY SERVICE IS MAINTAINED 
The fact that elected officials continue to dedicate substantial resources to constituency service well into their careers and after they have achieved high-status institutional positions is evidence that constituency services is a core function of congressional offices. This calls for renewed attention to the motivations for and effects of constituency service in modern U.S. democracy. As we collected and coded these data, we spoke to numerous staffers and agency officials. A recurring theme in the data and stories we heard were stories about constituency service casework interacting with other activities, including oversight investigations and even legislation. Conversely, new legislation often resulted in new forms of constituency service as legislators helped their constituents attain newly legislated benefits, deal with new paperwork requirements, or avoid new regulatory requirements. While constituency service may have underlying electoral motivations as formal models suggest, constituency services is also prominent yet an understudied form of legislator behavior in its own right. 

% REPRESENTATION 
Our finding that experience and institutional power allow legislators to do more policy work while maintaining or even increasing constituent service complements recent scholarship on representation.
The same legislators who \citet{Grose2011} and \citet{LowandeRitchieLauterbach2018} find doing more casework for protected groups also likely engage in higher rates of policy work on behalf of those groups, as well as higher rates of advocacy for nonprofits that serve those groups. 
While legislators must prioritize limited time, institutional power adds to the capacity of a legislative office as an institution to pursue both policy work and constituency service. Because institutional power comes with resources, representation matters not just in Congress but also in powerful positions like committee chairs. 
%% Devin: I re-read Mansbridge and decided that her categories (surprisingly) don't fit too neatly. Policy work is both promissory and anticipatory. Constituency service is also anticipatory and occasionally surrogate (surrogate is a very good term for constituency service, but Mansbridge defines it as representing people who are not in your district). I left it out for now unless you see a sensible way to say it. 

% INCUMBENCY ADVANTAGE 
Our results also offer a rationale for the incumbency advantage: when constituents choose between legislators, the choice to replace an incumbent legislator comes at the cost of less constituency service and policy work. Incumbents who have acquired experience and power in Washington are most able to deliver service for their constituents in the next legislative term.    

%% 2  Implications for policy (term limits, congressional staff, anything else)
\subsection{Implications for Policy}

% POTENTIAL IMPACT OF STAFF 
The large effects of legislator capacity that we find add to a recent wave of scholarship on the impact of congressional staffing. \citet{LaPira2020} show document many effects that decreasing levels of staffing may have on the functioning of Congress. Since increased staff for committee chairs is a likely mechanism for the capacity effects we find, this article offers a key outcome measure and effect sizes that may correspond to additional staff. As we note, committee chairs simultaneously obtain other forms of power like agenda control, but additional staff is a prominent form of increased capacity. To the extent that our results reflect the capacity boost of committee staff, they suggest that it should be possible to estimate the effects of increased staff. We cannot interpret the magnitude of our effects as the expected marginal value of additional staff in general, but our evidence is suggestive that congressional staff likely have measurable and potentially large effects on the volume of work that legislators have the capacity to do. 

% TERM LIMITS 
The findings from this paper also flip some of the concerns motivating proposed institutional reforms on their heads. For example, advocates of term limits often argue that elected officials lose touch with their district. In contrast, we show that the more experienced legislators provide more attention to their district, even as they take on more policy work. Rather than constituents lacking attention from experienced legislators, our results show that new legislators have less capacity to make requests to federal agencies. If constituents are concerned with the availability of service, then removing experienced legislators would decrease the amount of available service.   


%% 3 Future research (cite our other stuff, what else?) 

\subsection{Future Research}

Future research should further examine the mechanisms by which increasing experience and capacity shape legislator behavior. This could include explicit measures of office organization and efficiency and more nuanced measures of institutional power. This could also include measuring agency responsiveness to legislator requests. Likewise, future research could examine mechanisms related to shifting priorities. Finally, future work could include legislators' substantive areas of expertise. Does expertise increase a legislator's capacity to act in certain areas (e.g., certain agencies), leading to more capacity to do constituency service? Does increased institutional power lead legislators to develop expertise, for example, in certain committee work or specialized policy work that builds their capacity to influence certain agencies? 

The massive new dataset we introduce here will help scholars answer these questions and many others. With data from nearly all parts of the vast U.S. federal bureaucracy, future work can advance the study of descriptive representation, expanding on work by \citet{LowandeRitchieLauterbach2018}, who find that women, minority, and veteran members do more casework on behalf of groups that share their identity. The new data we collect will allow similar tests of representation for other demographic groups, including seniors, farmers, and low-income populations, to name just a few. Likewise, more data will allow new tests of prior work showing that members use lobbying the bureaucracy as a way to advance policy goals when they conflict with their party's agenda \citep{Ritchie2017}. Our systematic data allow tests of variation across policy domains and government functions. 

Critically, our systematic approach to data collection allows more general tests of legislator behavior. Any sample that focuses on a few policy domains or agencies will overrepresent legislators that sit on certain oversight committees and represent certain constituencies. Our near-census of legislator contacts minimizes such confounders and will allow researchers to test more general theories of legislator behavior, as we have done here.


\section{Conclusion} \label{s:conclude}

%Shifting priorities--the change in the ratio of policy work to constituency service--do not overpower the effects of increased capacity and power--the change in the overall number of requests to federal agencies, with even more powerful legislators focusing primarily on constituency service. 
Using a new data set on legislators' requests made to federal agencies, we show that as legislators gain experience and power, they both gain capacity and make more contacts with agencies and shift the balance of their contacts towards policy work. Consistent with the important role of capacity, we also show that legislators make fewer service requests at the start of their careers and that new legislators make substantially fewer service requests than their more experienced colleagues. 

%Theories that assert that capacity matters correctly predict that the levels of both constituency service and policy work increase as legislators gain experience and institutional power in Congress.
%Theories asserting that legislators shift priorities correctly predict that the ratio of policy work to constituency service increases as legislators spend more time and gain institutional power in Congress.
%Empirically, the net result of both effects is a net increase in the levels of both policy work and constituency service as legislators spend more time and gain institutional power in Congress; there is no tradeoff.

Taken together, our findings show that the reelection motivation remains a potent force for even the most powerful legislators in Washington. Rather than use increased power in the institution as an excuse to be less attentive to the district or increased time in Washington to be less interested in their constituencies, our results show that legislators with more power and experience provide no less constituency service to their district, even as they increase their attention to policy work. 

%TODO BETTER CONCLUSION
%Rather than forgetting about the district and contracting Potomac Fever, it would appear that more experience and power in Congress enables legislators to deliver particularistic goods to the district more effectively.  

\singlespacing

\bibliography{congress2019}

%\newpage
%\listoftodos[Notes]

\clearpage
\appendix
\setcounter{table}{0}
\renewcommand{\thetable}{A\arabic{table}}

\section*{Appendix}


\section{FOIA Data}
%
% Table created by stargazer v.5.2.2 by Marek Hlavac, Harvard University. E-mail: hlavac at fas.harvard.edu
% Date and time: Wed, Sep 09, 2020 - 20:10:39
\begin{table}[!htbp] \centering 
  \caption{Contacts From Members of Congress to Federal Agencies} 
  \label{responserates} 
\begin{tabular}{lccc} 
\\[-1.8ex]\hline \\[-1.8ex] 
Department & Components FOIAed & Records received &  N \\ 
\hline \\[-1.8ex] 
Agriculture & 29 & 29 &  9516 \\ 
Commerce & 19 & 18 &  8038 \\ 
Defense & 49 & 13 & 9739 \\ 
Education & 1 & 1 &  4689 \\ 
Energy & 8 & 2 &  6580 \\ 
Health and Human Services & 15 & 10 &  104145 \\ 
Homeland Security & 14 & 13 &  39633 \\ 
HUD & 2 & 1 &  33968 \\ 
Justice & 23 & 5 &  2611 \\ 
Labor & 22 & 12 &  53341 \\ 
State & 1 & 0 &  0 \\ 
Interior & 11 & 8 &  6079 \\ 
Treasury & 7 & 5 &  23869 \\ 
Transportation & 10 & 7 &  26787 \\ 
Veterans Affairs & 6 & 3 &  77842 \\ 
Independent Agencies & 77 & 47 &  81053 \\ 
\hline
Total & 294 & 174 &  487890 \\ 
\hline \\[-1.8ex] 
\end{tabular} 
\end{table} 


%%% Ellie: I manually changed the input code here so that the entire table would fit here.  
%\input{../tables/FOIA_response.tex}

% TODO replace with this one generated by Appendix.rmd, when I get the formatting right

\begin{tabular}{lrrr}
\toprule
Department & Components FOIAed & Records received & N\\
\midrule
Agriculture & 29 & 29 & 9603\\
Commerce & 19 & 18 & 7791\\
Defense & 49 & 13 & 9806\\
Education & 1 & 1 & 4676\\
Energy & 8 & 2 & 6256\\
\addlinespace
Health and Human Services & 15 & 10 & 109701\\
Homeland Security & 14 & 13 & 153151\\
Housing and Urban Development & 2 & 1 & 32158\\
Justice & 23 & 6 & 3096\\
Labor & 22 & 12 & 62353\\
\addlinespace
State & 1 & 0 & 0\\
Transportation & 10 & 7 & 26885\\
Veterans Affairs & 6 & 3 & 90808\\
the Interior & 11 & 8 & 6067\\
the Treasury & 7 & 5 & 23853\\
\addlinespace
Independent Agencies & 77 & 47 & 81152\\
\midrule
\textbf{Total} & \textbf{294} & \textbf{175} & \textbf{627356}\\
\bottomrule
\end{tabular}



\section{Contact Codebook} \label{a:codebook}
\singlespacing
%%% Note: From Ellie. I deleted everything here that we're not using in this paper.  

We provide the following codebook to a team of hand-coders to code each case of Congressional contact with federal agencies and extract information about the legislator. The codebook provides a series of steps to move from raw correspondence logs to data formatted for our analysis.  

\subsection{Congressional Correspondence Log Coding Guidelines}

The first step is to identify the columns that contain the member of Congress (or Committee), the date that the member-initiated correspondence, and the column that best describes the subject. These should be named FROM, DATE, and SUBJECT. 

We aim to classify the subject of correspondence between members of Congress and government agencies. You can do this using keywords (potential keywords in italics below) but may also require googling subject lines (e.g., what does this acronym mean in this context!?) and inferring why the legislator made the request. Doing so may require identifying a member's relevant policy positions. For example, if the subject is "mining regulations" or "open internet," a member's voting history on related bills or donations from the industry may help us infer if the letter was policy work on behalf of the industry (type 4) or not (type 5). Limiting your search to a date range around the letter date may yield relevant public statements. If you have questions, find something interesting, or, in your efforts to classify a confusing correspondence, you discover information like a related public statement, note it in the NOTES column. In some cases, columns other than the SUBJECT may offer helpful information. This may be difficult at first but will get easier. \\

The outcome is a spreadsheet with the first columns being FROM, DATE, SUBJECT, TYPE, CERTAINTY, ALT\_TYPE.\\


Below are five potential codes for the TYPE and three potential codes for your level of CERTAINTY that it is this type. If you are less than Very Certain (i.e., if only Fairly Certain, or Toss Up), also record your second best guess as ALT\_TYPE; otherwise, leave this column blank. Only leave NOTES if you think it would be helpful for the team to revisit the entry.

\subsubsection{TYPE}

1 = Personal Service\\

\hfill\begin{minipage}{\dimexpr\textwidth-2cm}
Definition: Individual, non-commercial constituent service.\\
Examples: Help with a government form, passport, visa, back pay, military honor, enlistment, criminal case, request for personal information (e.g., one’s FBI file), disability application, worker compensation, personal complaint, discrimination case, job application, health insurance, financial services complaints, etc.\\
\end{minipage}

2 = Commercial Service - Transactional \\

\hfill\begin{minipage}{\dimexpr\textwidth-2cm}
Definition: Anything related to a specific individual case by a business (including business owners like farmers and consultants).\\ 
General Examples: Help with a grant application, payment, loan, or contract (buying anything from or selling anything to a government agency). Help with an individual case of tax assessment, fine, or regulatory enforcement action. Help with public relations on behalf of a business.\\
Specific Examples: allocation of radio spectrum, a case against a company, tax dispute, contract for the purchase of military surplus, crop insurance distribution, debt settlement, foreclosure assistance, a fine for a law violation, etc. \\
\end{minipage}

3 = Government and Nonprofit Service - Transactional\\

\hfill\begin{minipage}{\dimexpr\textwidth-2cm}
Definition: Same as for (2-Commercial Service), but for municipal or state governments (including cities, counties, etc.) or non-business-oriented nonprofit organizations (i.e., NOT ones that represent an industry or trade association) \\
\end{minipage}

4 = Commercial Service - Policy \\

\hfill\begin{minipage}{\dimexpr\textwidth-2cm}
Definition: Anything applying to a class of commercial activity or businesses (e.g., shipping, airlines, agriculture), including legislation, bills, acts, appropriations, authorizations, etc. \\
General Examples: Authorization of or appropriation to a government program targeted towards a particular industry or industries. Regulation of industry or commercial practice or competition.\\
Specific Examples: Milk prices, insurance or loan eligibility criteria, purchasing policies, crop insurance rates, pollution criteria, classification of products for trade or taxation, conservation appropriation, worker visa types, restrictions, or caps, etc.\\
\end{minipage}
 
5 = Policy Work - NOT in the service of any individual, business, specific industry.\\

\hfill\begin{minipage}{\dimexpr\textwidth-2cm}
Examples of Policy Work: 
 \begin{tight_itemize} 
 \item Lawmaking 
\item Request for policy-relevant information. This includes prospective legislation, legislation under consideration, or already implemented legislation that requires oversight.  
\item Oversight
\item Committee requesting a report or testimony at a hearing
\item Requesting clarity on an agency rule
\item Lobbying administrative policy
\item Agency rulemaking with non-commercial implications (comments on agency rulemaking may often be (3)) 
\item Political work
\item Meeting with organized constituent groups (e.g., workers, people with disabilities, environmentalists) about policy (meetings with industry groups generally fall under (4)).
\item Media requests
 \end{tight_itemize} 
\end{minipage}
\bigskip


6 = Other \\

\hfill\begin{minipage}{\dimexpr\textwidth-2cm}
	Suggest a new category in the NOTES column, only if you cannot fit it under 1-4. For example, requesting dirt on one's political opponents could be called "partisan" as none of the above. Other specific types: thank you (for thank you notes with no other information), congratulations (for congratulatory correspondence on appointments or retirements with no other information), family member (for correspondence on behalf of a family member) \\
\end{minipage}

\clearpage


\section{Additional Models} \label{s:appendix_models}

\subsection{Constituency Service Only}

\begin{table}[hbt!]
\caption{The Effect Expierence and Institutional Power on Constituency Service} \label{t:models_con}
\begin{minipage}{\textwidth}
\begin{center}
\begin{tabular}[t]{lcccc}
\toprule
  & (1) & (2) & (3) & (4)\\
\midrule
Dependent Variable & Count & Count & Count & Log(Count+1)\\
\textbf{Committee Chair} & \textbf{\num{0.302}} & \textbf{\num{0.040}} & \textbf{\num{0.044}} & \textbf{\num{0.012}}\\
\midrule
 & (\num{0.108}) & (\num{0.064}) & (\num{0.064}) & (\num{0.007})\\
Ranking Member & \num{0.503} & \num{0.054} & \num{0.070} & \num{0.012}\\
 & (\num{0.108}) & (\num{0.067}) & (\num{0.067}) & (\num{0.007})\\
Prestige Committee & \num{0.321} & \num{0.031} & \num{0.025} & \num{0.013}\\
 & (\num{0.049}) & (\num{0.036}) & (\num{0.036}) & (\num{0.007})\\
First Year & \num{-0.138} & \num{-0.276} & \num{-0.265} & \num{-0.059}\\
 & (\num{0.040}) & (\num{0.055}) & (\num{0.054}) & (\num{0.008})\\
Second Year & \num{0.009} & \num{-0.128} & \num{-0.142} & \num{-0.019}\\
 & (\num{0.046}) & (\num{0.053}) & (\num{0.052}) & (\num{0.008})\\
Third Year & \num{0.030} & \num{-0.070} & \num{-0.088} & \num{-0.011}\\
 & (\num{0.047}) & (\num{0.047}) & (\num{0.046}) & (\num{0.007})\\
Fourth Year & \num{0.061} & \num{-0.055} & \num{-0.072} & \num{-0.006}\\
 & (\num{0.052}) & (\num{0.046}) & (\num{0.045}) & (\num{0.006})\\
Fifth Year & \num{0.001} & \num{-0.069} & \num{-0.064} & \num{-0.011}\\
 & (\num{0.044}) & (\num{0.034}) & (\num{0.033}) & (\num{0.005})\\
Sixth Year & \num{0.070} & \num{0.008} & \num{0.018} & \num{-0.004}\\
 & (\num{0.056}) & (\num{0.044}) & (\num{0.043}) & (\num{0.005})\\
Majority & \num{0.107} & \num{0.035} & \num{0.039} & \num{-0.005}\\
 & (\num{0.037}) & (\num{0.028}) & (\num{0.028}) & (\num{0.003})\\
President's Party & \num{0.051} & \num{0.031} & \num{0.033} & \num{0.009}\\
 & (\num{0.040}) & (\num{0.020}) & (\num{0.020}) & (\num{0.003})\\
\midrule
All Legislators & \checkmark & \checkmark &  & \checkmark\\
Served At Least 2nd Term &  &  & \checkmark & \\
Observations & \num{412111} & \num{412111} & \num{388997} & \num{412111}\\
Year x Agency FE & \checkmark & \checkmark & \checkmark & \checkmark\\
Legislator x Agency FE &  & \checkmark & \checkmark & \checkmark\\
\bottomrule
\multicolumn{5}{l}{\rule{0pt}{1em}\footnotesize Robust standard errors in parentheses, clustered by legislator.}\\
\end{tabular}
 % this one is from replication.rmd
\end{center}
\footnotetext{This table shows how the number of contacts hand-coded as constituecy service changes as legislators acquire more expiernece and power in Congress. Column 1 shows the average differences across committee assignments and years in Congress. Column 2 presents the difference-in-differences estimates. Column 3 subsets to legislators who serve at least 3 years in Congress. Column 4 takes the Log of the counts + 1 as the dependent variable.}
\end{minipage}
\end{table}

Table \ref{t:models_con} is identical to Table \ref{t:models_total} except that we subset the data to only legislator requests hand-coded as constituency service. 
Model 2 (Column 2 of Table \ref{t:models_con} and Figure \ref{f:m-con-predicted}) provide the estimated effects from the difference-in-differences specification in Equation \ref{e:diff1}. More experience increases the level of constituency service that legislators provide. The effect of being a committee chair is positive but not significant at the .05 level. We estimate that the experience gained between the first and second year in Congress causes an increase of 0.15 requests \textit{per agency}. The experience gained between the first and seventh years causes an increase of 0.28 per agency. Across all 90 agencies, this represents an increase of approximately 25 additional requests per year, 38.6\% of the average number of requests per year in our data. There is a smaller increase after the second year. The experience gained between the second and seventh year causes an increase of 0.13 per agency, an increase of approximately 12 additional requests per year, 38.6\% of the average number of requests per year in our data.


\begin{figure}[hbt!]
\centering
\caption{Predicted Number of Constituency Service Requests} \label{f:m-con-predicted}
\includegraphics[width = .49\textwidth]{figs/m-con-predicted-1}
\includegraphics[width = .49\textwidth]{figs/m-con-predicted-2}
\includegraphics[width = .49\textwidth]{figs/m-con-predicted-3}
\includegraphics[width = .49\textwidth]{figs/m-con-predicted-4}

\end{figure}

\subsection{Policy Work Only}

\begin{table}[hbt!]
\caption{The Effect Expierence and Institutional Power on Policy Work} \label{t:models_policy}
\begin{minipage}{\textwidth}
\begin{center}
\begin{table}
\centering
\begin{tabular}[t]{lcccc}
\toprule
  & 1 & 2 & 3 & 4\\
\midrule
Dependent Variable & Count & Count & Count & Log(Count+1)\\
Committee Chair & \num{0.203} & \num{0.158} & \num{0.159} & \num{0.036}\\
 & (\num{0.028}) & (\num{0.034}) & (\num{0.034}) & (\num{0.007})\\
Ranking Member & \num{0.144} & \num{0.090} & \num{0.092} & \num{0.025}\\
 & (\num{0.028}) & (\num{0.024}) & (\num{0.024}) & (\num{0.005})\\
Prestige Committee & \num{0.049} & \num{0.031} & \num{0.031} & \num{0.010}\\
 & (\num{0.010}) & (\num{0.010}) & (\num{0.010}) & (\num{0.003})\\
First Year & \num{-0.061} & \num{-0.076} & \num{-0.070} & \num{-0.030}\\
 & (\num{0.006}) & (\num{0.016}) & (\num{0.016}) & (\num{0.004})\\
Second Year & \num{-0.047} & \num{-0.042} & \num{-0.040} & \num{-0.018}\\
 & (\num{0.007}) & (\num{0.016}) & (\num{0.016}) & (\num{0.004})\\
Third Year & \num{-0.023} & \num{-0.031} & \num{-0.033} & \num{-0.013}\\
 & (\num{0.008}) & (\num{0.013}) & (\num{0.013}) & (\num{0.004})\\
Fourth Year & \num{-0.016} & \num{-0.011} & \num{-0.013} & \num{-0.006}\\
 & (\num{0.009}) & (\num{0.013}) & (\num{0.013}) & (\num{0.004})\\
Fifth Year & \num{-0.010} & \num{-0.009} & \num{-0.011} & \num{-0.006}\\
 & (\num{0.009}) & (\num{0.011}) & (\num{0.011}) & (\num{0.003})\\
Sixth Year & \num{-0.021} & \num{0.002} & \num{0.003} & \num{-0.006}\\
 & (\num{0.012}) & (\num{0.012}) & (\num{0.012}) & (\num{0.003})\\
Majority & \num{0.002} & \num{-0.001} & \num{0.000} & \num{-0.002}\\
 & (\num{0.007}) & (\num{0.006}) & (\num{0.006}) & (\num{0.002})\\
President's Party & \num{0.026} & \num{0.008} & \num{0.009} & \num{0.004}\\
 & (\num{0.009}) & (\num{0.006}) & (\num{0.006}) & (\num{0.001})\\
\midrule
All Legislators & ✓ & ✓ &  & ✓\\
Served At Least Second Term &  &  & ✓ & \\
Num.Obs. & \num{412111} & \num{412111} & \num{388997} & \num{412111}\\
Std.Errors & by: Legislator & by: Legislator & by: Legislator & by: Legislator\\
Legislator x Agency Fixed Effects &  & ✓ & ✓ & ✓\\
Year x Agency Fixed Effects &  & ✓ & ✓ & ✓\\
\bottomrule
\end{tabular}
\end{table}
 % this one is from replication.rmd
\end{center}
\footnotetext{This table shows how the number of hand-coded policy work contacts changes as legislators acquire more expiernece and power in Congress. Column 1 shows the average differences across committee assignments and years in Congress. Column 2 presents the difference-in-differences estimates. Column 3 subsets to legislators who serve at least 3 years in Congress. Column 4 takes the Log of the counts + 1 as the dependent variable.}
\end{minipage}
\end{table}


Table \ref{t:models_policy} is identical to Table \ref{t:models_total} except that we subset the data to only legislator requests hand-coded as policy work. 
Column 2 of Table \ref{t:models_policy} and Figure \ref{f:m-policy-predicted}) provide the estimated effects from the difference-in-differences specification in Equation \ref{e:diff1}. Across all measures of institutional power, we find that more power increases the level of policy work that legislators provide. Consider first the effect of being a committee chair. We estimate that becoming a committee chair causes an increase of 0.16 policy requests \textit{per agency} (95-percent confidence interval [0.09, 0.22]). Across all 90 agencies, this represents an increase of approximately 14 additional requests per year, 92.6\% of the average number of requests per year in our data. There is a smaller increase for individuals who become ranking members and those who join a Prestige Committee, though the increase is statistically significant for the prestige committee. Becoming a ranking member of a committee causes an increase of 0.09 contacts per agency while joining a prestige committee causes a 0.16 per agency increase in the number of contacts a member of Congress makes.

\begin{figure}[hbt!]
\centering
\caption{Predicted Number of Policy Requests} \label{f:m-policy-predicted}
\includegraphics[width = .48\textwidth]{figs/m-policy-predicted-1} 
\includegraphics[width = .48\textwidth]{figs/m-policy-predicted-2} 
\includegraphics[width = .48\textwidth]{figs/m-policy-predicted-3} 
\includegraphics[width = .48\textwidth]{figs/m-policy-predicted-4} 

\end{figure}



\end{document}







 



 


